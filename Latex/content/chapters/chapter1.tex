\chapter{Introduction}
\label{sec:intro}
In the present years the number of Cyber Physical Systems (CPSs) has drastically increased. This increase is mostly due to the emergence of the Internet of Things (IoT), as well as to the existence of several user friendly kits, auch as the RaspberryPi or the Arduino. With this recent trend new challenges are created in the monitoring and debugging of the above mentioned devices. This is especially difficult for applications, where the sensors are deployed in a hard to reach area. Moreover, sensors can often fulfill multi-purpose roles and might have to be reconfigured for different tasks.

In order to address both of these problems, SHAMPU(Single chip Host for Autonomous Mote Programming over USB)\cite{smeets2014demonstration} was developed at the University of Duisburg-Essen. SHAMPU is a design framework for monitoring and reprogramming WSNs. The main goal of SHAMPU is to be as small, lightweight and energy efficient as possible. Furthermore, a SHAMPU node is OS independent and can easily be attached to an existing node over USB. The SHAMPU nodes have multiple ways to transmit data between them, one of which is a wireless interface. In the current configuration, an ANT\cite{DynastreamInnovationsInc.2013} radio chip is used to transfer data between the SHAMPU nodes. Additionally, the nodes are connected to a base station which not only acts as a central data sink, but is also capable of pushing commands and data to the SHAMPU nodes. These wireless capabilities make it possible to remotely monitor, debug, and even reconfigure already deployed sensor nodes. 

In this thesis we evaluate the wireless capabilities of the SHAMPU framework. For that purpose we investigate use-cases for the different tasks the SHAMPU framework can perform: Program a device, collect data during operation, and interact with the system itself. For each of the use-cases we designed different experiments to assess how well ANT performs.
\newpage


\section{Related work}
\label{sec:related_work}
There are several different WSN test beds available. Each of these test beds is capable of monitoring and debugging an attached sensor node. The main difference between them is the size and specific role they were designed for. 

Some test beds like FlockLab \cite{Lim2013} allow to attach a wide array of different sensors and use a JTAG interface to be able to precisely capture debugging and timing information.

However, these types of test beds rely on a wired connection to interface with the test bed. Additionally, the platform itself has a large form factor. This makes it exceedingly difficult to test and debug already deployed WSNs since it might not be possible to easily get to the sensor in the planted location.

Other available solutions address this infrastructure dependency by attaching the test bed directly to the node and use a wireless connection to communicate with it. An example is Sensei-UU \cite{Rensfelt2009}, which uses a wireless 802.11 network. This set up allows the WSN to be tested and debugged while it is deployed. One drawback, however, is the power draw of 802.11 devices, which is huge compared to other technologies. To run the node itself on an external power source would not be a problem, but the use of a battery to power the node will hardly be feasible.

To address the power issue, it is possible to use a different technology which offers a lower power mode. Two examples are BTNodes \cite{Moser} or Smart-Its \cite{Kasten2000}, which use a Bluetooth connection. 
The structure of the network itself is similar to Sensei-UU: each node in the network has its own test bed attached and the test beds communicate with a central base station.

The main problem with Bluetooth is the limited size of the network, which makes it difficult to set up and use more complex network topologies. Also, Bluetooth communication is always 1:1 and does not allow broadcasts. The newest version of Bluetooth addresses the network size with the introduction of ScatterNets, but the setup and maintenance of the network remains challenging.

None of the above mentioned test beds exactly fit the design goals of SHAMPU. The test bed has to be directly attached to a fixed infrastructure, either because it has no wireless capabilities or because the power consumption is too high. BTNodes could be a possible alternative for SHAMPU. However Bluetooth connections are not well suited for the use in WSNs since they mostly use unicast.
