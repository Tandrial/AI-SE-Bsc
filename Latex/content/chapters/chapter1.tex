\chapter{Introduction}
\label{sec:intro}
In the current years the number of Cyber Physical Systems (CPSs) has drastically increased. This is mostly due to the emergence of the Internet of Things (IoT), as well as the existence of several user friendly kits, like the RaspberryPi or the Arduino. With this new trend new challenges are created in the monitoring and debugging of these devices. This is especially difficult for applications, where the sensors are deployed in a hard to reach area. In addition to this, the sensors can often fulfill a multi-purpose role and might need to be reconfigured for different tasks.

In order to address both these problems SHAMPU(Single chip Host for Autonomous Mote Programming over USB)\cite{smeets2014demonstration} was developed at the University Duisburg-Essen. SHAMPU is a design framework for monitoring and reprogramming WSN. The main goal of SHAMPU is to be as small, lightweight and energy efficient as possible. Furthermore a SHAMPU node itself is OS independent and can simple be attached to an existing node over USB. The SHAMPU nodes them self have multiple way to transmit data between them, one of which is a wireless interface. In the current configuration an ANT cite{DynastreamInnovationsInc.2013} radio chip is used to not only transfer data between the SHAMPU nodes. Additionally the nodes are also connected to a base station which not only acts as a central data sink, but is also able to push commands and data to the SHAMPU nodes. These wireless capabilities make it possible to remotely monitor, debug and even reconfigure an already deployed WSN. 

In this thesis we evaluate the wireless capabilities of the SHAMPU framework. For this purpose we utilize use-cases for the different tasks that the SHAMPU framework can perform: Program a device, collect data during operation and interact with the system itself. For each of the use cases we design different experiments to assess how well ANT performs in each use-case.
\newpage
\section{Related work}
\label{sec:related_work}
There are several different WSN test beds available. Each of these test beds are able to monitor and debug and attached sensor node. The main difference between them is the size and specific role they were designed for. 

Some test beds like FlockLab \cite{Lim2013} allow to attach a wide array of different sensors and use a JTAG interface to be able to precisely capture debugging and timing information.

However, these types of test beds rely on a wired connection to interface with the test bed itself. Additionally the platform itself has a large form factor. That makes it exceedingly difficult to test and debug already deployed WSN, since it might not be possible to easily get to the sensor in the planted location.

Other available solutions address this problem by attaching the test bed directly to the node and use a wireless connection to communicate with the test bed. An example for this is Sensei-UU \cite{Rensfelt2009} which uses a wireless 802.11 network. This allows the WSN to be tested and debugged while it is deployed. One drawback of this solution is the power draw of 802.11 devices, which is huge, compared to other technologies. If the node itself is run of an external power source that isn't a problem, but the use of a battery to power the node will hardly be feasible.

To address this power issue, it is possible to use a different technology which offers a lower power mode. Two examples for this are BTNodes \cite{Moser} or Smart-Its \cite{Kasten2000}, which use a Bluetooth connection. 
The structure of the network itself is similar to Sensei-UU: each node in the network has its own test bed attached and the test beds communicate with a central base station.

The main problem with Bluetooth is the limited size of the network, which makes it difficult to set up and use more complex network topologies. Also Bluetooth communication is always 1:1 and does not allow broadcasts. The newest version of Bluetooth addresses the network size with the introduction of ScatterNets, but the setup and maintenance of the network still remains challenging.

Non of the above mention test beds fit exactly the design goals of SHAMPU. The testbed is either has to be directly attached to a fixed infrastructure, either because it has no wireless capabilities or the power consumption is too high. BTNodes could be a possible alternative for SHAMPU, however Bluetooth connections are not well suited for the use in WSNs since they mostly use unicast.
