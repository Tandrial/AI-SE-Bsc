\chapter{Technical Background}
\section{SHAMPU}
SHAMPU (Single chip Host for Autonomous Mote Programming over USB) \cite{Smeets:2014:DAL:2602339.2602401} is a WSN testbed which allows remote debugging and reprogramming of sensor nodes. Its main advantages over other testbeds (see section~\ref{sec:related_work}) are portability, low cost, small size and low energy consumption. SHAMPU is used as an extension to an already existing sensor node. The only requirement is that the node provides an USB-Interface which is connected to SHAMPU. This single connection to the attached node makes SHAMPU completely OS independent.
\begin{figure}[h]
	\centering
	\includegraphics[scale=.5]{content/images/SHAMPUframework.png}
	\caption{Overview of the SHAMPU Framework \cite{Smeets:2014:DAL:2602339.2602401}}\label{fig:shampuframework}
\end{figure}
The SHAMPU framework (see Figure \ref{fig:shampuframework}) itself is split into multiple modules. The most important part for this thesis is the Code Reception Module, which allows the use of different protocols to connect to and communicate with the SHAMPU device. One option for the wireless communications with the SHAMPU device is the ANT protocol, on which we will focus in this evaluation.

\section{ANTAP1MxIB RF}
In order to use the ANT protocol each SHAMPU-mote is equipped with an ANTAP1MxIB RF Transceiver Module. The module was chosen because of its small form factor (20mm x 20mm) and its very low power draw. The ANTAP1M can handle up to 4 different ANT channels with a combined message rate which ranges between 0.5Hz and 200 Hz. 

\begin{figure}[h]
	\centering
	\includegraphics[scale=.5]{content/images/SHAMPUbase.JPG}
	\caption{SHAMPU base station}\label{fig:shampubase}
\end{figure}
In order to more easily test the capabilities of the ANT chip, we use a SHAMPU base station for all the experiments. This base station (see Figure \ref{fig:shampubase}) contains the ANTAP1MxIB and a serial interface, which allows the ANT to asynchronously communicate with a PC over RS-232. 

\section{ANT}
ANT \cite{DynastreamInnovationsInc.2013} is a wireless protocol which operates in the 2.4 GHz ISM Band. It was originally developed in 2003 by Dynastream Innovations Inc. for the use in wireless sensors. The ANT protocol is designed for the use in low power WSNs focusing on scalability and ease of use.

One of the advantages ANT has over other protocols, such as Bluetooth or ZigBee, is the high level of abstraction the ANT Protocol provides. 
\begin{figure}[h]
	\centering
	\includegraphics[scale=.5]{content/images/ANTstack.png}
	\caption{OSI-Layer vs. ANT Protocol\cite{Networks}}\label{fig:osilayer}
\end{figure}

This level of abstraction is achieved by incorporating the first 4 OSI-layers (see Figure \ref{fig:osilayer}) into the ANT protocol, thus allowing even low-cost MCU to set up and maintain complex wireless networks, since all the details of the communication are handled by the ANT-chip.

\subsection{ANT Topology}
In order for the ANT protocol to work each mote needs to be part of a network. As shown in figure \ref{fig:anttopo} the ANT protocol can be used to create simple or considerably more complex networks. Each mote inside a network is called an ANT node. In order for two nodes to communicate with each other they need to be connected via a channel.

\begin{figure}[h]
	\centering
	\includegraphics[scale=0.7]{content/images/ANTtopo.png}
	\caption{Example ANT Topologies\cite{DynastreamInnovationsInc.2013}}\label{fig:anttopo}
\end{figure}

\subsection{ANT Channels}
Most of the available channel types are bidirectional. The ANT protocol, however, still differentiates between master and slaves nodes. While a master node mostly sends data and a slave node mostly receives data, the slave retains the ability to respond to the incoming data.

The ANT protocol knows 125 different channels, each of them 1 MHz wide. Each Channel can support a data rate of 1 Mbps and up to 65533 nodes. To avoid interference between the channels isochronous self adjusting TDMA technology is used, which allows the ANT nodes to change the transmit timings as well as the frequency that is being used for the current channel. The use of the TDMA is completely handled by the ANT protocol, which means the user has no control over the process.

\begin{figure}[h]
	\centering
	\includegraphics[scale=0.7]{content/images/ANTsetup.png}
	\caption{ANT Channel Establishment\cite{DynastreamInnovationsInc.2013}}\label{fig:antsetup}
\end{figure}

Figure \ref{fig:antsetup} shows the procedure of opening up a channel between two ANT nodes. In the first step the Channel configuration is set, most important in this context being the channel type: 0x00 signifies that the node is a slave while 0x10 means that the node acts as a master. The next step is to set the Channel ID. Here the slave node needs to set the values of the master device in order to correctly connect. It is possible, however, to set wildcard values, which allows the slave to connect to any device sending on the same frequency. Before the final step, it is optionally possible to set the frequency of the channel and the message period. These steps are not mandatory though, since ANT defaults to fixed values for these: 2466 Hz transmission frequency and a message period of 8192. The last step constitutes of the opening of the channel itself, where care should be taken that the master opens the channel before it is opened by the slave device.

The ANT protocol distinguishes between two different Channel types:
\begin{description}
	\item{\textbf{Independent Channels}} \hfill \\ Independent Channels are used if there is only one node which transmits  data. There is no limit to the amount of slave devices which receive the messages that are being sent out. Furthermore, the message being sent out is broadcast to all nodes, it is not possible to address only specific nodes.
	\item{\textbf{Shared Channels}} \hfill \\ Shared Channels are used if there is more than one node which sends data. This type of channel is facilitated by the use of a Shared Channel Address, which in turn reduces the amount of data that can be transmitted at a time. All ANT nodes still receive every message, but only forward messages which have a matching address. The Channel master can decide to either use one or two bytes as the address, which allows for either 255 or 65535 slave devices in the same channel. 
\end{description}

\subsection{ANT Communication}

The ANT protocol supports three different data types: broadcast, acknowledge and burst. The data type is not part of the channel configuration, thus channels are able to use any combination of data types. The only exceptions are unidirectional channels, which can exclusively send broadcast data. The various data types differ in the way data is handled and transmitted.

\begin{itemize}
	\item{Broadcast data} \hfill \\ Broadcast data represents the most basic data type and, at the same time, the default. To start a broadcast transmission, the command needs to be issued just once, since the last sent packet is periodically resent as a broadcast. It does not matter if the last packet was part of a burst or acknowledge transmission, if no new data is available this packet is resent. Figure (MISSINGPIC) shows, how each broadcast transmission is aligned to the channel message period.	Since there is no answer from the receiving node, it is not possible to determine if the packet was transmitted correctly.
	
	\item{Acknowledge data} \hfill \\ Acknowledge data can be used to make sure a node has received a transmitted packet. After receiving an acknowledge packet the node will send a message back to the sender. Acknowledge data can only be used with bidirectional channels, yet both the master and the slave can use it. Figure (MISSINGPIC) shows that, just like broadcast data, acknowledge data is always aligned to a time slot, yet the answer from the receiver is sent immediately back to the sender.
	
	\item{Burst data} \hfill \\ Burst data provides a method to quickly transmit large amounts of data. This is achieved by ignoring the normal channel time slots and sending the packets immediately one after the other. This allows for a transmission rate of up to 20 kbps, which is much higher than other data transmission types. Similar to acknowledge data at the end of the transmission the sender is informed if the transfer failed or succeeded. The drawback of this method is, that burst data is prioritized over all other transmissions and will interrupt other transmissions over the same channel.
\end{itemize}

\subsection{ANT messages}
In the ANT protocol each message has the basic format as specified in Figure \ref{fig:antmsg}.
\begin{figure}[h]
	\centering
	\includegraphics[scale=.75]{content/images/ANTmsg.png}
	\caption{Ant message structure\cite{DynastreamInnovationsInc.2013}}\label{fig:antmsg}
\end{figure}
Each message starts with a special Sync-Byte and ends with a checksum, which is calculated by xoring all previous bytes. The Msg Length byte is the number of Message Content bytes. The Msg ID byte specifies which kind of data is contained in the message. The ANT protocol also provides an extended message format, which allows to attach further information to each message. The maximum length of the message content is 8 for each of the three data types.