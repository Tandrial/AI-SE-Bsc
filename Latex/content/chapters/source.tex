\chapter{Source Code}
\begin{verbatim}
#include <unistd.h>
#include <stdio.h>
#include <getopt.h>
#include "experiment.h"

char* ProgName; /* error.c */
extern uint16_t period;
extern FILE * fp;

int main(int argc, char** argv) {
	int option = -1;
	char deviceType = 0;
	int experimentNum = -1;
	char* port = "";
	
	ProgName = fileName(argv[0]); /* error.c */
	if (argc == 1) {
		error("Usage: %s -p /dev/ttyUSB[Number] -t [m(aster) or s(lave)] -n experiment [-f start_period]\n", ProgName);
	}
	
	while ((option = getopt(argc, argv, "p:t:n:f:")) != -1) {
		switch (option) {
			case 'p' :
			port = optarg;
			break;
			case 't' :
			deviceType = optarg[0];
			break;
			case 'n' :
			experimentNum = atoi(optarg);
			break;
			case 'f' :
			period = atoi(optarg);
			break;
			default:
			exit(EXIT_FAILURE);
		}
	}
	
	if (deviceType != 's' && deviceType != 'm') {
		error("Wrong deviceType! Must be 'm' or 's' is: %c\n", deviceType);
	}
	
	if (strncmp(port, "/dev/ttyUSB", 11)) {
		error("Port information wrong!");
	}
	
	fp = fopen("result.txt", "a");
	printf("Port = %s\nType = %c\nMessage Period = %d (%f Hz)\n", port, deviceType, period, 32768.0 / period);
	initANT(port);
	setTransmitPower(ANT_TRANSMIT_POWER_0DBM);
	
	switch (experimentNum) {
		case 1:
		printf("Experiment 1: Broadcast Data Transfer between two nodes\n");
		doExperiment1(deviceType);
		break;
		
		case 2:
		printf("Experiment 2: Broadcast Data Transfer between multiple nodes\n");
		doExperiment2(deviceType);
		break;
		
		case 3:
		printf("Experiment 3: Ackowledge Data Delay\n");
		doExperiment3(deviceType);
		break;
		
		case 4:
		printf("Experiment 4: Ackowledge Data Transfer between two nodes\n");
		doExperiment4(deviceType);
		break;
		
		case 5:
		printf("Experiment 5: Communication Distance\n");
		doExperiment5(deviceType);
		break;
		
		case 6:
		printf("Experiment 6: Burst Data Transfer between two nodes\n");
		doExperiment6(deviceType);
		break;
		
		default:
		printf("Available Experiments:\n");
		printf("Experiment 1: Broadcast Data Transfer between two nodes\n");
		printf("Experiment 2: Broadcast Data Transfer between multiple nodes\n");
		printf("Experiment 3: Ackowledge Data Delay\n");
		printf("Experiment 4: Ackowledge Data Transfer between two nodes\n");
		printf("Experiment 5: Communication Distance\n");
		printf("Experiment 6: Burst Data Transfer between two nodes\n");
		error("%d is not a valid experiment! \n", experimentNum);
		
		break;
	}
	flushBuffer();
	fclose(fp);
	return EXIT_SUCCESS;
}

\end{verbatim}