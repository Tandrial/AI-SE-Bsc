\addchap{Eidesstattliche Erklärung}
Hiermit versichere ich, dass ich die vorliegende Arbeit ohne Hilfe Dritter und nur mit den angegebenen Quellen und
Hilfsmitteln angefertigt habe. Ich habe alle Stellen, die ich aus den Quellen wörtlich oder inhaltlich entnommen habe,
als solche kenntlich gemacht. Diese Arbeit hat in gleicher oder ähnlicher Form noch keiner Prüfungsbehörde vorgelegen.\\
\\
\\
\\
\ort, am \datum

%Erläuterung: Diese eidesstattliche Erklärung ist für Abschlussarbeiten
%durch alle Prüfungsordnungen verbindlich vorgeschrieben, muss also in
%der Arbeit enthalten sein, entweder wie hier am Anfang oder alternativ
%ganz am Ende der Arbeit. Für andere schriftliche Arbeiten gilt sie natürlich
%ebenfalls. Studierenden wird deswegen empfohlen, diese Erklärung allen
%schriftlichen Ausarbeitungen beizufügen. Beachten sie bitte auch die Aus-
%führungen zum Thema "Plagiatismus und gute wissenschaftliche Praxis"