\documentclass[a4]{exam}
\usepackage[utf8]{inputenc}
\usepackage{amsmath}
\usepackage{amssymb}
\usepackage{algorithm}
\usepackage{algorithmicx}
\usepackage{algpseudocode}

\renewcommand{\author}{Michael Krane}
\newcommand{\gruppe}{Gruppe E2}
\newcommand{\assignment}{Übungsblätter }
\newcommand{\class}{BeKo 2013/14}

\pagestyle{headandfoot}
\firstpageheadrule
\runningheadrule
\header{\author}{\gruppe : \assignment}{\class}
\firstpagefooter{}{}{Seite \thepage\ von \numpages}
\runningfooter{}{}{Seite \thepage\ von \numpages}

\qformat{\underline{\textbf{Aufgabe \thequestion}} \thequestiontitle\hfill }

\begin{document}
\begin{questions}

%
% Aufgabe 1
%

\titledquestion {Chomsky-Hierachie}
\begin{parts}

% Aufgabe 1a
\part Zunächst gilt Typ-3 $\subseteq$ Typ-2 $\subseteq$ Typ-1 $\subseteq$ Typ-0 \\\\
Typ-0 Grammatiken: Eine Grammatik G ist vom Typ-0, wenn keine Einschränkungen vorliegen. \\ \\
Typ-1 Grammatiken: Eine Grammatik G heißt kontextsensitiv (Typ-1), wenn alle Produktions-\\regeln die folgende Form haben: $\alpha_1 A \alpha_2 \rightarrow \alpha_1 \beta \alpha_2 $ mit $ A \in V, \alpha_1, \alpha_2 \in (V  \cup \Sigma)^* , \beta \in (V  \cup \Sigma)^+$ \\ \\
Typ-2 Grammatiken: Eine Grammatik G heißt kontextfrei (Typ-2), wenn alle Produktionsregeln folgende Form haben: $A \rightarrow \beta $ mit $ A \in V, \beta \in (V  \cup \Sigma)^+$\\ \\
Typ-3 Grammatiken: Eine Grammatik G heißt regulär (Typ-3), wenn alle Produktionsregeln folgende Form haben: $A \rightarrow aB $ oder $A \rightarrow a$  mit $ A,B \in V, a \in \Sigma$ \\ 

% Aufgabe 1b
\part \begin{itemize}
\item $ L_1 $ ist eine Typ-2 Grammatik, da es mit Typ-3 nicht möglich ist, mehr als ein Nichtterminalsymbol pro Ableitungsschritt hinzuzufügen. Dies ist allerdings nötig, damit die Anzahl der as und bs gleichbliebt.
\item $L_2$ ist eine Typ-3 Grammatik, da die Anzahl der as und bs nicht voneinander abhängt. Deswegen reicht es, wenn man pro Ableitungsschritt ein Terminalsymbol hinzufügt wird.
\item $L_3$ ist eine Typ-1 Grammatik, da es mindestens eine Produktionsregel mit der Form  $\alpha_1 A \alpha_2 \rightarrow \alpha_1 \beta \alpha_2 $ geben muss,  um die gleiche Anzahl von terminal Symbolen an unterschiedlichen Stellen zu erzeugen.\\
\end{itemize}

% Aufgabe 1c
\part $G_1 = (V, \Sigma, P, S)$ \\
$V = \{S,A,B\}$ \\
$\Sigma = \{a,b\}$\\
$P = \{ S \rightarrow aAb, A \rightarrow aAb \mid B, B \rightarrow bBa \mid ba\}$ \\
Da alle Produktionsregeln die Form $A \rightarrow \beta $ mit $ A \in V, \beta \in (V  \cup \Sigma)^+$ haben, ist  die Grammatik vom Typ-2.\\
\end{parts}

%
% Aufgabe 2
%
\titledquestion{Wiederholung: Chomsky-Grammatiken}
\begin{parts}

% Aufgabe 2a
\part $ L_1 = \{ a^{n}b^m \mid n,m \in \mathbb{N} , n >1 \} $

% Aufgabe 2b
\part $ L_2 = \{ a^{2n}b^{3n} \mid n \in \mathbb{N}  \} $

% Aufgabe 2c
\part aabbb $ \in L_1 \cap L_2 $ 

% Aufgabe 2d
\part $ L_1 \cup L_2 = \{ a^{2n}b^{3n} \mid n \in \mathbb{N}  \} $ da $ L_2 \subseteq  L_1$\\
\end{parts}

%
% Aufgabe 3
%
\titledquestion{Einführung Turingmaschine}
\begin{parts}

% Aufgabe 3a
\part Im Binärsystem ist die Multiklikation mit 2 nichts anderes als eine Verschiebung der Bits um eine Stelle nach links.
\begin{enumerate}
\item Gehe nach rechts an das Ende der Zahl.
\item Schreibe eine '0' rechts neben die Zahl.
\item Gehe nach links bis zum Anfang der Zahl und wechsle in den Endzustand.
\end{enumerate}

% Aufgabe 3b
\part Im Binärsystem ist die Multiklikation mit 4 eine Linksverschiebung um 2 Stellen. Entweder man führt die Turingmaschine aus (a) zweimal hintereinander aus oder man modifiziert die Turingmaschine aus (a) und führt den zweiten Schritt zweimal hintereinander aus.\\\\
\end{parts}

%
% Aufgabe 4
%
\titledquestion{Mengen, Funktionen und Relationen}
\begin{parts}

% Aufgabe 4a
\part 
\begin{enumerate}
\item $\{ n \in \mathbb{N} \mid n \mod 4 = 0  \wedge  n \mod 100 = 0 \implies n \mod 400 = 0 \} $ (Schaltjahre)
\item $\{(n,m) \in \mathbb{Z}  \times \mathbb{Z} \mid m \not= 0 \} $ \\
\end{enumerate}

% Aufgabe 4b
\part  $\{1,2,3\} \cup \{x,z\} = \{1,2,3,x,z\}$ \\
$\{1,2,3\} \cap \{x,z\} = \varnothing$ \\
$\{1,2,3\} \setminus \{x,z\} = \{1,2,3\}$ \\
$\{1,2,3\} \setminus \{2\} = \{1,3\}$ \\
$\{1,2,3\} \times \{x,z\} = \{(1,x),(1,z),(2,x),(2,z),(3,x),(3,z)\}$ \\
$\mathcal{P} (\{\varnothing\}) = \{\varnothing,\{\varnothing\}\}$ \\
$\mathcal{P} (\mathcal{P} (\varnothing)) = \{\varnothing,\{\varnothing\},\{\{\varnothing\}\},\{\varnothing,\{\varnothing\}\}\}$ \\

% Aufgabe 4c
\part 
\begin{enumerate}
\item $R_f = \{(x,y) \in X \times Y \mid  f(x) = y\}$
\item Eine Funktion $f_R : X \rightarrow Y$ legt für jedes $ x \in X$ genau ein $y \in Y$ fest. Es gibt allerdings Relationen, wo mehrere y einem x zugewiesen werden:  \\ $R_f =  \{(x,y) \in X \times Y \mid x^2 + y^2 = 2^2\} = \{\dots ,(2,2),(2,-2),\dots \} $
\item Es ist möglich jede Relation $R$ als eine Funktion $g_R : X \rightarrow \mathcal{P}(Y)$ darzustellen, da durch eine Potenzmenge die Probleme der mehrfachen Zuweisung ($x \in X $ wird auf eine Menge mit Mächtigkeit $> 1$ abgebildet) und der fehlenden Zuweisung ($x \in X$ wird auf die leere Menge $\varnothing$ abgebildet) gelöst werden.
\end{enumerate}
\end{parts}

%
% Aufgabe 5
%
\titledquestion {Zählen von Turingmaschinen}
\begin{parts}

% Aufgabe 5a
\part Für die deterministische TM gibt es ingesamt 18 verschiedene "rechte Seiten" der Übergangsfunktion ($|\{z_1,z_2,z_3\}| * |\{a,\Box\}|* |\{R,L,N\}| = 18$) und 6 verschiedene "linke Seiten" der Übergangsfunktion ($|\{z_1,z_2,z_3\}| * |\{a,\Box\}|= 6$) , Also gibt es ingesamt $18^6 = 34012224$ mögliche Übergangsfunktionen.

% Aufgabe 5b
\part Für die nichtdeterministische TM gibt es ingesamt $2^{18}$ verschiedene "rechte Seiten" der Übergangsfunktion ($|\mathcal{P}(\{z_a,z_b\}\times |\{x,y,\Box\} \times\{R,L,N\})| = 2^{18}$) und 6 verschiedene "linke Seiten" der Übergangsfunktion  ($|\{z_a,z_b\}| * |\{x,y,\Box\}|= 6$). Es gibt also ingesamt ${2^{18}}^6 = 262144^{6} = 3,245 * 10^{32}$ mögliche Übergangsfunktionen.

% Aufgabe 5c
\part Die Überlegung aus (a) kann in eine verallgemeinerte Formel umgewandelt werden: Die Anzahl der Übergangsfunktionen kann mit 
$ (3 * m * n)^{m*n} $  berechnet werden, wobei $ m = |\Gamma| $ und $ n =  |Z|$.

% Aufgabe 5d
\part Die Überlegung aus (b) kann auch wieder in eine verallgemeinerte Formel umgewandelt werden: Die Anzahl der Übergangsfunktionen kann mit:
$2^{(3mn)^{m*n}} = 2^{3m^2n^2}$ berechnet werden, wobei $ m = |\Gamma| $ und $ n =  |Z|$.
\end{parts}

%
% Aufgabe 6
%
\titledquestion{Was macht diese Turingmaschine?}
\begin{parts}

% Aufgabe 6a
\part Für die Eingabe 'ab' durchläuft die TM folgende Schritte: \\
$z_0 ab \vdash 
az_0b \vdash 
abz_0\Box \vdash 
a z_sb \#  \vdash 
a \#z_b\#  \vdash
a \#\#z_b\Box \vdash
a \#z_r\#b \vdash
a z_s\#\#b \vdash
z_sa\#\#b \vdash
\#z_a\#\#b  \vdash \\
\#\#z_a\#b \vdash 
\#\#\#z_ab \vdash
\#\#\#bz_a\Box \vdash
\#\#\#z_rba \vdash
\#\#z_r\#ba \vdash
\#z_s\#\#ba \vdash
z_s\#\#\#ba\vdash
z_s\Box \#\#\#ba \vdash
z_d\#\#\#ba \vdash
z_d\#\#ba \vdash
z_d\#ba \vdash
z_dba \vdash
z_Eba $ \\

Für die Eingabe 'ab' : $z_0ab \vdash^* z_Eba$\\
Für die Eingabe 'aabb': $z_0aabb \vdash^* z_Ebbaa$.

% Aufgabe 6b
\part Die TM schreibt die Eingabe rückwärts auf das Band und löscht danach die ursprüngliche Eingabe.

% Aufgabe 6c
\part $z_0$: Es wird das rechte Ende der Eingabe mit einem \#-Zeichen makiert. Danach wechselt die TM nach $z_s$.\\
$z_s$: Es wird vom rechten Ende des Wortes aus nach einem a oder b gesucht, welches mit \# überschrieben wird. Bei einem a geht es mit $z_a$ weiter, bei einem b geht es mit $z_b$ weiter. Falls es kein a oder b mehr gibt, wechselt die Turingmaschine nach  $z_d$.\\
$z_a$: Das gefunde a wird an das rechte Wortende geschrieben. Danach wechselt die TM nach $z_r$.\\
$z_b$: Das gefunde b wird an das rechte Wortende geschrieben. Danach wechselt die TM nach $z_r$.\\
$z_r$: Der Lesekopf wird wieder auf das ürsprüngliche Ende der Eingabe gesetzt, dann wechselt die TM nach $z_s$.\\
$z_d$: Der Lesekopf wird auf das erste Zeichen des umgedrehten Wortes gesetzt und löscht dabei alle erzeugen \#.  Danach TM wechselt in den Endzustand $z_e$.\\
$z_E$: Die TM befindet sich im Endzustand und terminiert.

\end{parts}

%
% Aufgabe 7
%
\titledquestion{Turingmaschine für eine reguläre Sprache}
\begin{parts}

% Aufgabe 7a
\part $G = (V, \Sigma, P, S)$ \\
$V = \{S,A, B\}$ \\
$\Sigma = \{a,b\}$\\
$P = \{ S \rightarrow bS \mid aA , A \rightarrow aA \mid ab\}$ \\

% Aufgabe 7b
\part Man kann $aa^*a$ in $aaa^*$ umformen. Damit ist $ L: b^* aaa^* b $. \\\\$
\delta(z_0,a) = (z_1, \Box, R) $ $ z_0$ erkennt $b^*$ und wechselt bei einem a nach $z_1$ .\\ $
\delta(z_0,b) = (z_0, \Box, R) \\
\delta(z_0,1) = (z_4, \Box, R) \\
\delta(z_0,\Box) = (z_E, \Box, N) \\\\
\delta(z_1,a) = (z_2, \Box, R) $  $z_1$ erkennt das zweite a und wechselt nach  $z_2$. Ansonsten in den\\$ 
\delta(z_1,b) = (z_4, \Box, R) $  Fehlerzustand $z_4$. \\ $
\delta(z_1,1) = (z_4, \Box, R) \\
\delta(z_1,\Box) = (z_E, \Box, N) \\\\
\delta(z_2,a) = (z_2, \Box, R)$ $z_2$ erkennt $a^*$, wechselt bei einem b nach $z_3$ und sonst in den Fehlerzustand $z_4$. \\ $ 
\delta(z_2,b) = (z_3, \Box, R) \\
\delta(z_2,1) = (z_4, \Box, R) \\
\delta(z_2,\Box) = (z_E, \Box, N) \\\\ 
\delta(z_3,a) = (z_4, \Box, R) $ Falls das Band leer ist, wird eine '1' geschrieben und in den Endzustand \\$
\delta(z_3,b) = (z_4, \Box, R) $   gewechselt, ansonsten in den Fehlerzustand $z_4$.\\ $
\delta(z_3,1) = (z_4, \Box, R) \\
\delta(z_3,\Box) = (z_E, 1, N) \\\\ 
\delta(z_4,a) = (z_4, \Box, R) $ Fehlerzustand. Überschreibt Band mit $\Box$ und wechselt in den Endzustand.\\ $
\delta(z_4,b) = (z_4, \Box, R) \\
\delta(z_4,1) = (z_4, \Box, R) \\
\delta(z_4,\Box) = (z_E, \Box, N) \\\\
\delta(z_E,a) = (z_E, a, N)  $ Endzustand. Übergangsfunktion vorgegeben.  \\ $
\delta(z_E,b) = (z_E, b, N) \\
\delta(z_E,1) = (z_E, 1, N) \\
\delta(z_E,\Box) = (z_E, \Box, N) \\ 
$
\end{parts}

%
% Aufgabe 8
%
\titledquestion{Wiederholung: Funktionen Teil II }
\begin{parts}

% Aufgabe 8a
\part $f_a : \mathbb{R} \rightarrow \mathbb{R}, f_a(x) := 2 $ \\ Da  $f_a(1) = f_a(2) $ aber $ 1 \not= 2$ also ist $f_a$ nicht injektiv. \\ Nicht surjektiv da es z.B. für  $y \in \mathbb{R} = 3$  kein $ x \in \mathbb{R}$ mit $ f_b(x) = 3$ gibt.

\newpage

% Aufgabe 8b
\part $f_b : \mathbb{N} \rightarrow \mathbb{R}, f_b := x $ \\  Aus $ f_b(a) = f_b(b)  $ folgt $ a  = b$ also ist $f_b(x)$ injektiv. \\  Nicht surjektiv da es z.B. für $y \in \mathbb{R} = 2,5$  es  kein $ x \in \mathbb{N}$ mit $ f_b(x) = 2,5$ gibt.

% Aufgabe 8c
\part $f_c : \mathbb{R} \rightarrow \{2\}, f_c(x) := 2 $ \\ Da  $f_c(1) = f_c(2) $ aber $ 1 \not= 2$ also ist $f_c$ nicht injektiv. \\ $f_c$ ist surjektiv, da es für alle $ y \in \{2\}$ ein Urbild in $\mathbb{R}$ gibt.

% Aufgabe 8d
\part $f_d : \mathbb{R} \rightarrow \mathbb{R}, f_d(x) := x $ \\ Aus $ f_d(a) = f_d(b)  $ folgt $ a  = b$ also ist $f_d(x)$ injektiv. \\
Für alle $y \in \mathbb{R} = Y$ existiert ein $x \in \mathbb{R} = X$, falls $ X = Y$ und $f_d(x) = x$ also die Identität  von x ist.
\end{parts}

%
% Aufgabe 9
%
\titledquestion {Überabzahlbarkeit und Diagonalisierung} 
\begin{parts}

% Aufgabe 9a
\part Wenn $I_1$ abzählbar ist, kann man mit der Funktion $f_n : \mathbb{N}_0 \Rightarrow I_1 $, wobei $f_n(x)$ die x. Dezimalstelle der n-ten Zahl angibt eine Liste erstellen, die alle Zahlen $\in I_1$ enthält. \\
\begin{tabular}{r| r r r r r r r}
n & $f_n(0) $ & $f_n(1) $ &$f_n(2) $ &$f_n(3) $ &$f_n(4) $ & $\dots$ \\ \hline
0& -0, & 1 & 2 & 3 & 4\\
1& 0, & 1 & 2 & 3 & 4\\
2& -0, & 2 & 3 & 4 & 5\\
3& 0, & 2 & 3 & 4 & 5\\
\vdots \\
\end{tabular}

Wenn man nun eine neue Zahl $\in I_1$ mit  $g:\mathbb{N}_0 \Rightarrow  \mathbb{N}_0$ mit $g(n) = \begin{cases} 0 &$falls $ n = 0 \wedge f_0(0) = -0 \\ -0 &$falls $ n = 0 \wedge f_0(0) = 0 \\ 1 & $falls $ f_n(n) = 2 \\    2 & $falls $ f_n(n) \not= 2 \\    \end{cases} $ definieren. Da die neue Zahl, per Definition nicht in der Liste enthalten seien kann, muss $I_1$ überabzählbar groß sein.

% Aufgabe 9b
\part $\varphi_r: I_r \Rightarrow I_1$ , $ \varphi_r(x) = \frac{x}{r} $ mit $\varphi^{-1} : I_1 \Rightarrow I_r$ , $\varphi^{-1}(x) = rx$

% Aufgabe 9c
\part Die Bildmenge der Funktion $f(x) = \tan(x)$ auf dem Intervall $\left]-\frac{\pi}{2}, \frac{\pi}{2}\right[$ ist $\mathbb{R}$. Man kann für jedes $r \in \mathbb{R}$ eine Funktion $f_r : I_r \Rightarrow \mathbb{R} ,f_r(x) = \tan (\frac{x}{r}*\frac{\pi}{2}) $ definieren. $f_r$ ist bijektiv da die Umkehrfunktion $f_r^{-1}: I_r \Rightarrow \mathbb{R}, f_r^{-1}(x) = \frac{2r}{\pi}\arctan(x)$ existiert.

% Aufgabe 9d
\part Es gibt genau dann eine bijektive Funktion $g_n : \mathbb{N}^n \Rightarrow \mathbb{R}$, wenn $\mathbb{N}^n$ und $\mathbb{R}$ gleich mächtig sind. Mit vollständiger Induktion kann bewiesen werden, das alle n-Tupel aus $\mathbb{N}$ gleichmächtig und damit abzählbar sind: Für $\mathbb{N}^2$ gibt es eine Funktion  $f_2 : \mathbb{N} \times \mathbb{N} \Rightarrow \mathbb{N}_0$ , $f_2(n,m) = \left(\displaystyle\sum\limits_{i=0}^{m+n} i\right) +n$ Für $n=3$ gibt es eine rekursive Funktion $f_3 : \mathbb{N} \times \mathbb{N} \times \mathbb{N} \Rightarrow \mathbb{N}_0$, $f_3(m,n,k) = f_2(m,f_2(n,k))$. Für $ n \ge 3$ kann diese Funktion zu $f_n : \mathbb{N}^n \Rightarrow \mathbb{N}_0$, $f_n(m_1,\dots,m_n) = f_2(m_1,f_{n-1}(m_2,\dots,m_n))$ verallgemeinert werden.
 Da die Menge der reellen Zahlen aber überabzählbar ist, sind $\mathbb{N}^n$ und $\mathbb{R}$ nicht gleichmächtig, daraus folgt das es keine bijektive Funtion $g_n : \mathbb{N}^n \Rightarrow \mathbb{R}$ geben kann.
\end{parts}

%
% Aufgabe 10
%
\titledquestion {Bandbeschränkt vs. Zeitbeschränkt} 
\begin{parts}

% Aufgabe 10a
\part Stimmt nur, wenn alle $b(n)$ besuchtet Stellen links oder rechts von der Startposition liegen und der Kopf niemals eine Bwegung in die andere Richtung macht, d.h. jede Stelle wird nur genau einmal besucht.

% Aufgabe 10b
\part Stimmt. Mit $t(n)$ Schritten kann sich der Lesekopf maximal $t(n)$-mal bewegen. Geht der Kopf immer nur in eine Richtung können maximal $t(n) +1$ Stellen besucht werden. $t(n)$ Stellen rechts/links von der Startposition, plus die Startposition selber. Alle anderen Bewegungenskombinationen haben $ <b(n)$ verschiedene besuchte Stellen, da mindestens eine Stellen doppelt besucht werden muss.

% Aufgabe 10c
\part Stimmt nicht. Da das Band unendlich lang ist, ist es möglich das die TM die ganze Zeit über nur nach links oder rechts läuft und sich deswegen keine Konfiguration wiederholt.

% Aufgabe 10d
\part Stimmt. Da es bei einer bandbeschränkter TM nur endlich viel Platz gibt und das Arbeitsalphabet eine endliche Menge ist, gibt es auch auch nur endlich viele Permutationen für die Konfiguration einer  bandbeschränkten TM. Da aber unendlich viele Schritte gemacht werden, muss mindestens eine Konfiguration mehrmals erreicht werden.

% Aufgabe 10e
\part Stimmt. Wenn eine Konfiguration vor erreichen des Endzustandens mehrfach erreicht wird, bebeutet dass das die TM in einer Endlosschleife und kann nicht terminieren. Sie führt also unendlich viele Schritte aus.
\end{parts}


%
% Aufgabe 11
%
\titledquestion {Zweiband-Turingmaschine} 
Sei $ M = (\{z_1,z_E\},\{a,b\},\{a,b,\Box\},\delta,z_1,\Box,\{z_E\})$  eine Mehrband-Turingmaschine mit \\$\delta : Z \times \Gamma^k \Rightarrow Z \times  \Gamma^k \times \{L,R,N\}^k$ und $ k = 2$ Bändern. \\\\ 
 In $z_1$ wird die Eingabe von Band 1 von links nach rechts gelesen und nacheinander mit $\Box$ überschrieben. Gleichzeitig wird das gelesene Zeichen von nach rechts nach links auf das Ausgabeband geschrieben. Dadurch wird die Eingabe umgedreht. Sobald die Eingabe komplett abgearbeitet ist wechselt die TM in den Endzustand $z_E$\\$
\delta(z_1,(a,a)) = (z_1,(a,a),(N,N)) \\
\delta(z_1,(a,b)) = (z_1,(a,b),(N,N))\\
\delta(z_1,(a,\Box)) = (z_1,(\Box,a),(R,L))\\ 
\delta(z_1,(b,a)) = (z_1,(b,a),(N,N))\\
\delta(z_1,(b,b)) = (z_1,(b,b),(N,N))\\
\delta(z_1,(b,\Box)) = (z_1,(\Box,b),(R,L))\\ 
\delta(z_1,(\Box,a)) = (z_1, (\Box,a), (N,N))\\
\delta(z_1,(\Box,b)) = (z_1, (\Box,b), (N,N))\\
\delta(z_1,(\Box,\Box)) = (z_E, (\Box,\Box ), (N,N))$\\\\
$z_E$ ist der Endzustand. Hier passiert nichts. \\$
\delta(z_E,(a,a)) = (z_E,(a,a),(N,N)) \\
\delta(z_E,(a,b)) = (z_E,(a,b),(N,N))\\
\delta(z_E,(a,\Box)) = (z_E,(a\Box),(N,N))\\ 
\delta(z_E,(b,a)) = (z_E,(b,a),(N,N))\\
\delta(z_E,(b,b)) = (z_E,(b,b),(N,N))\\
\delta(z_E,(b,\Box)) = (z_E,(\Box,b),(N,N))\\ 
\delta(z_E,(\Box,a)) = (z_E, (\Box,a), (N,N))\\
\delta(z_E,(\Box,b)) = (z_E, (\Box,b), (N,N))\\
\delta(z_E,(\Box,\Box)) = (z_E, (\Box,\Box), (N,N))
$

%
% Aufgabe 12
%
\titledquestion {Lineare Bandbeschrankung }
Damit eine $c*n$ lange Bandbeschriftung auf eine Beschriftung der Länge $n$ umgebaut werde kann, muss die Anzahl der Informationen pro Zellen erhöht werden. Dazu fasst man jeweils $c$-Zeichen zu einem neuen Zeichen zusammen, z. B. mit $c = 2$ und der Eingabe: $\Box \mid a\mid b \mid b \mid a \mid a \mid a \mid \Box \Rightarrow \Box a \mid bb \mid aa \mid a\Box$. Das ürsprüngliche Bandalphabet  $(\Gamma = \{a,b,\Box\})$ wird zu $\Gamma^\prime = \{aa,ab,a\Box,ba,bb,b\Box,\Box a,\Box b, \Box\Box\}$ umgebaut. Das Bandalphabet vergrößert sich also um einen Faktor von $ \frac{ \mid \Gamma^\prime\mid}{\mid \Gamma \mid} = 3 = 3^{2-1}$. Für den allgemeinen Fall vergrößert sich das ursprüngliche Bandalphabet also um den Faktor $ \mid\Gamma\mid^{c-1}$ .

%
% Aufgabe 13
%
\titledquestion {LOOP-Programme}
LOOP-Programm A berechnet $x_2 - x_1$, indem es $x_1$-mal 1 von $x_2$ abzieht. \\
LOOP-Programm B berechnet $x_1 - x_2$, indem es $x_2$-mal 1 von $x_1$ abzieht. \\
LOOP-Programm C berechnet $\lvert x_1 - x_2 \rvert$. Indem es zuerst Programm B ausführt und das Ergebnis in $x_3$ speichert, für $x_2 \ge x_1$ ist $x_3 = 0$. Danach wird das Programm A ausgeführt und das Ergebnis in $x_4$ gespeichert, wobei $x_4 = 0$ falls $x_1 \ge x_2$ ist. Da nun entweder $x_3 = 0$ oder $x_4 = 0$ ist, kann man einfach $x_3 + x_4$ rechnen um die positive Differenz von $x_1$ und $x_2$ zu erhalten.\\

%
% Aufgabe 14
%
\titledquestion {GOTO-Programme}
\begin{parts}

% Aufgabe 14a
\part  $f_1: \mathbb{N}_0 \times \mathbb{N}_0 \Rightarrow \mathbb{N}_0$, $f_1(a,b) = a+ b$
\begin{align}
&\phantom{M_0:} x_0 = x_1; &\text{Das Ergebnis $x_0$ = a gesetzt.}\notag\\
&M_1: \text{IF } x_2 = 0  \text{ THEN GOTO }  M_2;&\text{Fall $x_2 = 0$ ist muss nichts addiert werden.}  \notag\\
&\phantom{M_0:}x_0 = x_0 + 1;&\text{Das Ergebnis $x_0$ wird um Eins erhöht.} \notag\\
&\phantom{M_0:}x_2 = x_2 - 1;&\text{Die Eingabe $x_2$ wird um Eins reduziert.} \notag\\
&\phantom{M_0:}\text{GOTO } M_1; & \text{Sprung zum Check für  $x_2$ = 0.} \notag\\
&M_2: \text{HALT};\notag&
\end{align}

% Aufgabe 14b
\part  $f_2: \mathbb{N}_0 \times \mathbb{N}_0 \Rightarrow \mathbb{N}_0$, $f_2(a,b) = a - b$
\begin{align}
&\phantom{M_0:} x_0 = x_1; &\text{Die Ausgabe $x_0$ = a gesetzt.}\notag\\
& M_1: \text{IF } x_2 = 0 \text{ THEN GOTO } M_2; &\text{Falls $x_2 = 0$ ist muss nichts subtraiert werden.}\notag\\
&\phantom{M_0:}x_0 = x_0 -1; &\text{Das Ergebnis $x_0$ wird um Eins reduziert.}\notag\\
&\phantom{M_0:}x_2 = x_2 -1; &\text{Die Eingabe $x_2$ wird um Eins reduziert.}\notag\\
&\phantom{M_0:}\text{IF } x_0  = 0 \text{ THEN GOTO } M_2; &\text{Abbruch, da es keien Zahlen $< 0$ gibt.}\notag\\
&\phantom{M_0:}\text{GOTO }M_1; &\text{Sprung zum Check für  $x_2$ = 0.}\notag\\
&M_2:\text{HALT};\notag
\end{align}

% Aufgabe 14c
\part  $f_3: \mathbb{N}_0 \times \mathbb{N}_0 \Rightarrow \mathbb{N}_0$, $f_3(a,b) = min(a,b)$
\begin{align}
&\phantom{M_0:} x_3 = f_2(x_1,x_2);&\text{Die Differenz von $x_1$ und $x_2$ wird gebildet.}\notag\\
&\phantom{M_0:}\text{IF } x_3 = 0 \text{ THEN GOTO } M_1;&\text{Wenn $x_3 = 0$ ist, ist $x_2 >= x_1$, also mit $M_1$ weiter.}\notag\\
&\phantom{M_0:}x_0 = x_2;&\text{$x_2$ ist die kleinere Zahl.}\notag\\
&\phantom{M_0:}\text{GOTO } M_2;&\text{Sprung zum Programmende.}\notag\\
&M_1: x_0 = x_1;&\text{$x_1$ ist die kleinere Zahl.}\notag\\
&M_2: \text{HALT}; \notag
\end{align}

% Aufgabe 14d
\part  $f_4: \mathbb{N}_0 \times \mathbb{N}_0 \Rightarrow \mathbb{N}_0$, $f_4(a,b) = a $ MOD $ b$
\begin{align}
&M_1: x_0 = f_2(x_1,x_2);&\text{Die Differenz von $x_1$ und $x_2$ wird gebildet.}\notag\\
&\phantom{M_0:}\text{IF } x_0 = 0 \text{ THEN GOTO } M_2;&\text{Abbruchbedingung. Differenz = 0.}\notag\\
&\phantom{M_0:}x_1 = x_0;&\text{Die Differenz $(> x_2)$  ist das neue a.}\notag\\
&\phantom{M_0:}\text{GOTO } M_1;&\text{Sprung zurück.}\notag\\
&M_2: x_4 = f_2(x_1,x_2)&\text{Differenz $x_1$ und $x_2$.}\notag\\
&\phantom{M_0:} x_5= f_2(x_2,x_1)&\text{Differenz $x_2$ und $x_1$.}\notag\\
&\phantom{M_0:}\text{IF } x_4 = 0 \text{ THEN GOTO } M_3;&\text{Falls $ x_1 - x_2 = 0$ ist ...}\notag\\
&\phantom{M_0:}\text{GOTO } M_5;&\notag\\
&M_3:\text{IF } x_5 = 0 \text{ THEN GOTO } M_4;&\text{und $x_2 - x_1 = 0  \Rightarrow x_1 = x_2$}\notag\\
&\phantom{M_0:}x_0 = x_1;&\text{Falls $x_1 \ne x_2$ steht in  $x_1$ der modulo ...}\notag\\
&\phantom{M_0:}\text{GOTO } M_5;&\text{}\notag\\
&M_4:x_0 = 0;&\text{ansonsten geht der Modulo glatt auf $(=0)$.}\notag\\
&M_5:\text{HALT};&\notag
\end{align}

% Aufgabe 14e
\part  $f_5: \mathbb{N}_0 \times \mathbb{N}_0 \Rightarrow \mathbb{N}_0$, $f_5(a,b) = a $ DIV $ b$
\begin{align}
&M_1:\text{IF } x_1 = 0 \text{ THEN GOTO } M_3;&\text{Abbruchbedingung: 0 div a ist immer 0.}\notag\\
&\phantom{M_0:} x_3 = f_4(x1,x2); &\text{$x_3$ enthält $x_1$ mod $x_2$.}\notag\\
&\phantom{M_0:} x_3 = f_2(x1,x3); &\text{Damit die Divsion glatt wird $x_3 = x_1 - x_3$ }\notag\\
&M_2:\text{IF } x_3 = 0 \text{ THEN GOTO } M_3;&\text{Abbruchbedingung: $x_2$ passt nicht in $x_1$ .}\notag\\
&\phantom{M_0:} x_3 = f_2(x1,x2); &\text{Division durch wiederholte Subtraktion.}\notag\\
&\phantom{M_0:}x_0 = x_0 + 1;&\text{$x_0$ zählt mit wie oft $x_2$ in $x_1$ passt.}\notag\\
&\phantom{M_0:}x_1 = x_3;&\text{Minuend aktualisieren.}\notag\\
&\phantom{M_0:}\text{GOTO } M_1;&\text{Sprung zum Check für  $x_3$ = 0.}\notag\\
&M_3: \text{HALT};\notag
\end{align}
\end{parts}

%
% Aufgabe 15
%
\titledquestion {WHILE-Programme}
\begin{parts}

% Aufgabe 15a
\part  Für $x_1 =3 $ und $x_2 = 2$ gibt das Programm $x_0 = 12$ aus.

% Aufgabe 16b
\part Das Programm implementiert die Funktion $f : \mathbb{N} \Rightarrow \mathbb{N}$, $ f(x_1,x_2) = 2^{x_2}*x_1$. Wobei die innere Schleife (Zeile 3-6) die Zahl verdoppelt und die aüßere Schleife (Zeile 1-8) die Anzahl der Verdoppelungen vorgibt.

\end{parts}

%
% Aufgabe 16
%
\titledquestion {WHILE-Programm} 
Das folgende WHILE-Programm berechnet $f(x_1,x_2) = ggT(x_1,x_2)$:
\begin{align}
&\text{WHILE } x_1 \ne x_2 \text{ DO} & \notag \\
&\phantom{WHILE }\text{IF } x_2 - x_1  = 0\text{ THEN} &\notag\\
&\phantom{WHILE IF} x_1:=  x_1-x_2 &\notag\\
&\phantom{WHILE }\text{ELSE } &\notag\\
&\phantom{WHILE IF} x_2 := x_2-x_1& \notag \\
&\phantom{WHILE }\text{END;} &\notag\\
&\text{END}; &\notag\\
&x_0 := x_1&\notag
\end{align}

Makro $\text{WHILE } x_1 \ne x_2 \text{ DO A END} $:
\begin{align}
&\text{IF } x_2 = x_1\text{ THEN } x_3 := 0 \text{ ELSE } x_3 := 1 \text{ END;}& \notag \\
&\text{WHILE }x_3 \ne 0 \text{ DO}& \notag \\
&\phantom{WHILE } A;&\notag\\
&\phantom{WHILE }\text{IF } x_2 = x_1\text{ THEN } x_3 := 0 \text{ ELSE } x_3 := 1 \text{ END;}& \notag \\
&\text{END}&\notag
\end{align}

Makro $ \text{IF } x_2 = x_1\text{ THEN } A \text{ ELSE B END} $:
\begin{align}
& x_3 := x_1 - x_2;& \notag \\
& x_4 := x_2 - x_1;& \notag \\
& x_3 := x_3 + x_4;& \notag \\
&\text{IF } x_3 = 0\text{ THEN } A \text{ ELSE } B \text{ END}& \notag 
\end{align}

Makro $\text{IF } x_1 = 0 \text{ THEN A ELSE B END}$:
\begin{align}
&x_2 := x_1; & \notag \\
&x_3 := 1; & \notag \\
&x_4 := 0; & \notag \\
& \text{WHILE } x_2 \ne 0 \text{ DO} &\notag\\
&\phantom{WHILE } x_2 := x_2- 1;& \notag \\
&\phantom{WHILE } x_3 := 0;& \notag \\
&\phantom{WHILE } x_4 := 1& \notag \\
&\text{END;} &\notag\\
& \text{WHILE } x_3 \ne 0 \text{ DO} &\notag\\
&\phantom{WHILE } x_3 := x_3- 1;& \notag \\
&\phantom{WHILE } \text{A}& \notag \\
&\text{END;} &\notag\\
& \text{WHILE } x_4 \ne 0 \text{ DO} &\notag\\
&\phantom{WHILE } x_4 := x_4- 1;& \notag \\
&\phantom{WHILE } \text{B}& \notag \\
&\text{END} &\notag
\end{align}

%
% Aufgabe 17
%
\titledquestion {LOOP-Programm}
\begin{parts}

% Aufgabe 17a
\part In einem LOOP-Programm wird die Anzahl der Schleifendurchläufe festgelegt, bevor die Schleife beginnt. Innerhalb der Schleife kann die Anzahl dann nicht mehr verändert werden, deshalb muss jede Schleife und damit auch jedes LOOP-Programm terminieren.

% Aufgabe 17b
\part Damit eine Programmiersparche turingmächtig ist, muss sie partielle Funktionen berechnen können. Da alle LOOP-Programme terminieren kann es keine partiellen Funktionen geben, also sind LOOP-Programme nicht turingmächtig.

% Aufgabe 17c
\part Das folgende LOOP-Programm berechnet die Funktion $f : \mathbb{N}_0 \Rightarrow \mathbb{N}_0$, $f(n) = \begin{cases} 0 &$, falls $ n = 0  \\ \lceil \log_2(n) \rceil& $, sonst $ \\  \end{cases}$
\begin{align}
& x_2 := 1; & \notag \\
&\text{LOOP } x_1  \text{ DO } &\notag\\
&\phantom{LOOP} \text{IF} (x_2 +1) - x_1 = 0 \text{ THEN}& \notag \\
&\phantom{LOOP LOOP} x_2 := 2 * x_2; & \notag \\
&\phantom{LOOP LOOP} x_3 := x_3 +1;& \notag \\
&\phantom{LOOP} \text{END} & \notag \\
&\text{END} &\notag\\
&x_0 := x_3;&\notag
\end{align}
\end{parts}

%
% Aufgabe 18
%
\titledquestion {primitiv-rekursive Funktionen} 
\begin{parts}

% Aufgabe 18a
\part  
$f_1(0,y) = g(y) = id = \pi^1_1(y)$ \\
$f_1(x+1,y) = h(f_1(x,y),x,y) = dec(\pi^3_1(f_1(x,y),x,y))$

% Aufgabe 18b
\part 
$\chi_{\{0\}}(0) = 1$ \\
$\chi_{\{0\}}(n+1) =  h(\chi_{\{0\}}(n),n) = 0$ 

% Aufgabe 18c
\part 
$f_2(0,y) = g(y) = f_3(y)$\\
$f_2(x+1,y) = h(f_2(x,y),x,y) =  y * \pi^3_1(f_2(x,y),x,y)$

$f_3(0) = 0$\\
$f_3(x+1) = h(f_3(x),x) = 1$

\end{parts}

%
% Aufgabe 19
%
\titledquestion {SKIP-Programm} 
\begin{parts}

% Aufgabe 19a
\part Damit ein GOTO-Programm durch ein SKIP-Programm simuliert werden kann, muss die komplette GOTO-Syntax durch die SKIP-Syntax ausgedrückt werden:
\begin{itemize}
\item Wertzuweisung: Die Syntax für die Wertezuweisung ist bei beiden Sprachen identisch.
\item Unbedingter Sprung: $M_j : \text{ GOTO } M_i =  \begin{cases} \text{SKIP } (i-j-1) & $, falls $ i > j \\  \text{GOTOSTART; SKIP} (j-1) & $, falls $ i = j \\ \text{GOTOSTART; SKIP}(j-i-1) & $, falls $ i < j \\  \end{cases}$  
\item Bedingter Sprung: Die Syntax ist fast identisch, nur GOTO  muss durch SKIP ersetzt werden.
\item Stopanweisung: HALT = SKIP(n); , wobei n = Anzahl der Zeilen im GOTO-Programm.
\end{itemize}

% Aufgabe 19b
\part Damit ein SKIP-Programm durch ein GOTO-Programm simuliert werden kann, muss die komplette SKIP-Syntax durch die GOTO-Syntax ausgedrückt werden:
\begin{itemize}
\item Wertzuweisung: Die Syntax für die Wertezuweisung ist bei beiden Sprachen identisch.
\item Unbedingter Sprung: $\text{SKIP}(k) = \begin{cases} M_i : \text{GOTO } M_{i+k+1}; & $, falls $ i+k+1 < n \\  M_i: \text{HALT;} & $, falls $ i+k+1 \geq n  \end{cases}$ n = Zeilenanzahl.
\item Bedingter Sprung: Die Syntax ist fast identisch, nur SKIP  muss durch GOTO ersetzt werden.
\item Sprung zum Start: GOTOSTART = $\text{GOTO } M_1$;
\end{itemize}

\end{parts}

%
% Aufgabe 20
%
\titledquestion {Totale Funktionen} 
Laut Aufgabe ist die Menge der gültigen Programme $L$ rekursiv aufzählbar. Das bedeutet das es eine surjektive Funktion $F : \mathbb{N}_0 \Rightarrow L$ gibt, die einer Zahl n das n-te Programm in der Aufzählung zuordnet. Wenn man nun eine neues Programm definieren, mit $h: \mathbb{N}_0 \Rightarrow \mathbb{N}_0$, $h(x) = F(x) + 1$, welches die Ausgabe des x-ten Programmes + 1 berechnet. Diese Funktion ist laut Definition berechenbar und total, muss also in der  Menge der gültigen Programme L sein. Sei jetzt i der Index von dem Programm $h(x)$ in der Aufzählung $L$. Dann würde $F(i) = h(i) = F(i) +1$ gelten. Das führt allerdings zu einem Wiederspruch und daher kann $h(x)$ nicht Teil der Aufzählung sein. Das wiederspricht der 2. Eigenschaft des Sprache (jede total Funktion kann mit einem Programm der Sprache beschrieben werden), weshalb die Sprache so nicht existieren kann.
%
% Aufgabe 21
%
\titledquestion {primitiv-rekursive Funktionen}
 $\chi_{\{0\}} : \mathbb{N}_0 \Rightarrow \mathbb{N}_0$, $f(n) = \begin{cases} 1 &$, $ x = 0  \\ 0 &$, $x > 0 \\  \end{cases}$  siehe Blatt 5 Aufgabe 18.
\begin{parts}
% Aufgabe 21a
\part  
$leq(x,y) = \chi_{\{0\}}(x-y)$

% Aufgabe 21b
\part 
$geq(x,y) = \chi_{\{0\}}(y-x)$

% Aufgabe 21c
\part
$eq(x,y) = mult(leq(x,y), geq (x,y))$

\end{parts}

%
% Aufgabe 22
%
\titledquestion {$\mu$-rekursive Funktionen}
\begin{parts}

% Aufgabe 22a
\part  Mit dem $\mu$-Operator ist $\mu f_1 : \mathbb{N}^1_0 \Rightarrow \mathbb{N}_0$, $\mu f_1(y) = min \{n \mid f_1(n,y) = 0\}$, mit $f_1(x,y) = y-x$. $f_1$ ist genau dann = 0, wenn $ y-n = 0 \Leftrightarrow y = n$ ist. \\
Also ist $\mu f_1 : \mathbb{N}^1_0 \Rightarrow \mathbb{N}_0$, $\mu f_1(x) = x$, also die Identitätsfunktion.

% Aufgabe 22b
\part 
\begin{enumerate}
\item $\sqrt[x]{y} = n \Leftrightarrow y \leq n^x \Leftrightarrow y - n^x \leq 0$. Die Funktion bricht also ab, sobald $n^x \geq y$ ist.\\ $n^x$ ist primitiv-rekursiv, siehe Blatt 5 Aufgabe 18(c).\\
$\mu f_{pot} : \mathbb{N}^2_0 \Rightarrow \mathbb{N}_0$, $\mu f_{pot}(x,y) = min \{n \mid f_{pot}(n,x,y) = 0\}$ \\
$f_{pot} : \mathbb{N}^3_0 \Rightarrow \mathbb{N}_0$, $f_{pot}(n,x,y) = y - n^x$\\



\item Für $n = x$ ist $1-eq(x,n) = 0$ und $1-eq(y,n) =1$, also $(1-eq(x,n)) *(1-eq(y,n)) = 0$. Für $n = y$ analog. Da n hochgezählt wird tritt wird $f_{min}(n,x,y)$ zuerst bei dem kleineren Parameter $=0$.\\
$\mu f_{min} : \mathbb{N}^2_0 \Rightarrow \mathbb{N}_0$, $\mu f_{min}(x,y) = min \{n \mid f_{min}(n,x,y)= 0\}$ \\
$f_{min} : \mathbb{N}^3_0 \Rightarrow \mathbb{N}_0$, $f_{min}(n,x,y) = (1-eq(x,n)) * (1-eq(y,n))$ 
\end{enumerate}

\end{parts}

%
% Aufgabe 23
%
\titledquestion {Reduktionen I}
Mit der  Vorverarbeitung $f(a,b,c)$ ist es möglich mit der Maschine $M_{sum}$ zuentscheiden ob $c= a- b$ gilt: \\
$f: \mathbb{Z}^3 \Rightarrow \mathbb{Z}^3$, $f(a,b,c) = (a,-b,c)$, also $c' = c$ , $b' = -b$ und $a' = a$.

%
% Aufgabe 24
%
\titledquestion {Reduktionen II}
\begin{parts}

% Aufgabe 24a
\part  Für alle Zahlen gilt: gerade Zahl + 1 = ungerade Zahl. Also kann man mit der Vorverarbeitung $f_a$ das Problem A auf B reduzieren:
$f_a: \mathbb{Z} \Rightarrow \mathbb{Z}$, $f_a(x) = x +1$.

% Aufgabe 24b
\part Wenn $x \mod 11 = 0$ dann ist $x = 11n$, also ein Vielfaches von 11. Desweiteren ist $ y = 11x$, also ein vielfaches von 121 und damit $y \mod 121 = 0$. Also kann mit $f_b$ Problem A auf B reduziert werden: 
$f_b: \mathbb{Z} \Rightarrow \mathbb{Z}$, $f_b(x) = 11x$.

% Aufgabe 24c
\part 
Man kann ein Wort $ w \in A $ in ein Wort $w' \in B$ mit folgenden Regeln umbauen:\\
$a \Rightarrow c$ : Jedes a wird durch ein c ersetzt. \\
$b \Rightarrow de$ : Jedes b wird durch de ersetzt.\\
Damit ist $\#_c(w') = \#_a(w)$, $\#_d(w') = \#_b(w)$ und $\#_e(w') = \#_b(w)$ \\ also $2*\#_b(w)  = \#_d(w') + \#_e(w')$ und damit ist das Kriterium für Sprache B erfüllt. 

\end{parts}

%
% Aufgabe 25
%
\titledquestion {Entscheidbarkeit}
\begin{parts}

% Aufgabe 25a
\part Man kann mit der Funktion $f$ eine TM bauen, die charakteristische Funktion $\chi_L$ berechnet. Dabei geht die TM folgendermaße vor: 
\begin{enumerate}
\item Die Eingabe ist $w \in \sum^*$.
\item Setze einen Counter $ i = 0$.
\item Berechne f(i).
\item Falls $f(i)=w$, lösche das Band schreibe ein 1 und wechsel in den Endzustand.
\item Setze $i = i+1$ und gehe nach 3. solange eine Eigenschaft erfüllt ist:
\begin{enumerate}
\item $|w| < |f(i)|$
\item $|w| = |f(i)|$ und es gibt ein $ j \in \{1, \dots ,|w|\}$, sodass $ \forall (k < j \in \mathbb{N} \mid  w_k = f(i)_k \wedge w_j <_{\sum} f(i)_j) $
\end{enumerate}
\item Lösche das Band, schreibe eine 0 und wechsel in den Endzustand.
\end{enumerate}

% Aufgabe 25b
\part Man kann mit $TM_{\chi_L}$  eine TM bauen die $f$ berechnet, dabei geht die TM folgendermaßen vor:
\begin{enumerate}
\item Die Eingabe ist $i \in \mathbb{N}_0$
\item Setze zwei Counter $j = 0$ und $l = 0$
\item Simuliere auf $TM_{\chi_L}$ alle Wörter mit Länge l in längen-lexikographischer Reihenfolge. Für jedes Wort $\in L $ erhöhe j um 1.
\item Falls j = i breche ab und gebe das letze erzeugte Wort zurück.
\item Setze $ l = l +1$ und gehe nach 3.
\end{enumerate}
\end{parts}

%
% Aufgabe 26
%
\titledquestion {Franz der Frisör (Diagonalisierung reloaded)}


%
% Aufgabe 27
%
\titledquestion {Reduktionen}


%
% Aufgabe 28
%
\titledquestion {Ein neues Halteproblem}


%
% Aufgabe 29
%
\titledquestion {Abgeschlossenheit bezüglich Reduktionen}
\begin{parts}

% Aufgabe 29a
\part  

% Aufgabe 29b
\part 

\end{parts}

%
% Aufgabe 30
%
\titledquestion {Satz von Rice / Unentscheidbarkeit }
Es ist zuzeigen, dass $L_0 \le L_= $ gilt. Falls $L_0 \le  L_=$ gilt, kann man aus $L_=$ eine TM für $L_0$ bauen.
Dies ist mit der Vorverarbeitungs Funktion $f(w) = w\#w_0$ möglich, wobei $w_0$ die Kodierung einer TM ist, die die konstante Nullfunktion berechnet. Da laut Satz von Rice $L_0$ 
unentscheidbar ist und $L_0 \le L_=$ gilt, ist auch $L_=$ unentscheidbar.

%
% Aufgabe 31
%
\titledquestion {Postsches Korrespondenzproblem }
\begin{parts}
% Aufgabe 31a
\part  Eine mögliche Lösung ist : 3,1,4 
\begin{align}
&ab \mid aba  \mid abaa  \notag &&\\
&a  \phantom{b}\mid ab \phantom{a} \mid abaa  \notag &&
\end{align}

% Aufgabe 31b
\part Das MPCP gilt als gelöst, wenn die komplette Sequenz gleich ist, wenn man mit dem Paar $(x_1,y_1)$ beginnt, unterscheiden sich die Sequenzen bereits im 1. Zeichen.

% Aufgabe 31c
\part
\begin{enumerate}[1]
  \item Es gibt ingesamt $K^n$ verschiedenen Indexfolgen. Da man in jedem Schritt eins der K-Wortpaare auswählen kann.
  \item 
	\begin{align}
\sum\limits_{n=1}^{N} \lambda(K,n) = \sum\limits_{n=1}^{N} K^n &= K + K^2 + \dots + K^N \notag \\
	s_n &= K + K^2 + \dots + K^N  & \mid&  * K \\
	K s_n &= K^2 + K^3 + \dots + K^{N+1} \\
	\text{(1) - (2): }  s_n - K s_n &= K - K^{N+1} \notag\\
	s_n(1-K) &= K - K^{N+1} &\mid& / (1-K) \notag \\
	s_n &= \frac{K - K^{N+1}}{1-K} \notag \\
	\sum\limits_{n=1}^{N} \lambda(K,n) &= \frac{K - K^{N+1}}{1-K} \notag
	\end{align}
\end{enumerate}

\end{parts}

%
% Aufgabe 32
%
\titledquestion {Reduktion H $\leq$ MPCP}
\begin{parts}

% Aufgabe 32a
\part  Die Turingmaschine sucht vom Wortanfang aus ein b. Wenn die TM ein b gefunden hat wird der Kopf wieder auf den Start der Eingabe gesetzt und die Maschine wechselt in den Endzustand. Die TM terminiert bei allen Eingaben die mindestens ein b enthalten : $L(M) =  \{a^*b^+a^*b^*\}$

% Aufgabe 32b
\part Da es sich um ein MPCP handelt ist $(x_1,y_1) = (\#,\#z_0b\#)$ 
\begin{enumerate}
  \item Kopierregeln = $\{(a,a),(b,b),(\Box,\Box),(\#,\#)\}$

  \item Überführugnsregeln = $\{(z_0b,z_1b), (z_E\Box,z_E\Box), (z_Ea,z_Ea), (z_Eb,z_Eb),\\
	(z_0\Box,\Box z_0), (z_0a,az_0), (z_1\Box,\Box z_E)\\
	(az_1a,z_1aa), (bz_1a,z_1ba), (\Box z_1a,z_1\Box a),\\
	(az_1b,z_1ab), (bz_1b,z_1bb), (\Box z_1b,z_1\Box b),\\
	(\#z_1a,\#z_1\Box a), (\#z_1b,\#z_1\Box b),\\
	(z_E\#,z_E\Box\#),\\
	(z_0\#,\Box z_0\#), (z_1\#,z_E\Box\#)$

  \item Löschregeln = $\{(az_E,z_E),(z_Ea,z_E),(bz_E,z_E),(z_Eb,z_E),(\Box z_E,z_E),(z_E\Box ,z_E)\}$

  \item Abschlussregeln = $\{(z_E\#\#,\#)\}$
\end{enumerate}

% Aufgabe 32c
\part Für die Eingabe 'b' durchläuft die TM folgende Konfigurationen:
$z_0 b \vdash 
z_1b \vdash 
z_1\Box b \vdash 
 z_Eb $

% Aufgabe 32d
\part $(\#,\# z_0b\#),(z_0b,z_1b),(\# z_1b,\# z_1\Box b),(\#,\#),(z_1\Box, \Box z_E),(b,b),(\#,\#),\\(\Box z_E, z_E), (b,b),(\#,\#)(z_Eb,z_E),(\#,\#)(z_E\#\#,\#)$
\end{parts}

\newpage 

%
% Aufgabe 33
%
\titledquestion {Es weihnachtet sehr! }
Ein dem Angebot wird eine Turingmaschine beschrieben. Das Band wird durch die Häuser simuliert, das Bandalphabet ist 0 (kein Geschenk im Haus) und 1 (Geschenk im Haus). Jeder Wichtel   stellt eine Übergangsfunktion oder eine Abfolge von Übergangsfunktionen (Der Schlittenfahrer) dar. Das Problem ist der letze Punkt des Vertrages. Es soll ein Computerprogramm geben das entscheidet ob die Wichtel zurück kommen oder nicht. Wenn man das Angebot als eine Art Turingmaschine interpretiert, würde diese Programm das Halteproblem lösen. Das Halteprogramm ist allerdings nicht entscheidbar und deshalb kann es ein solchen Computerprogramm nicht geben. Herr M. E. Phisto ist also ein Betrüger.

%
% Aufgabe 34
%
\titledquestion {Eigenschaften von Turingmaschinen}
\begin{parts}

% Aufgabe 34a
\part  Entscheidbar. Es ist möglich die TM für 100 Schritte zu simulieren und dann zuüberprüfungen ob die TM sich in einem Endzustand befindet oder nicht.

% Aufgabe 34b
\part Entscheidbar. Man kann die TM simulieren und dabei die durchlaufenen Konfigurationen speichern. Dann tritt irgendwann einer der drei möglichen Fälle ein :
\begin{itemize}
  \item Die Konfiguration der TM ist länger als 100 Zellen. $\Rightarrow$ Abbruch, Aussgabe falsch.
  \item Die TM terminiert mit einer Konfiguration $< 100$ Zellen. $\Rightarrow$ Aussage korrekt.
  \item Die TM terminiert nicht (= eine Konfiguration kommt mehrmals vor), aber alle Konfigurationen sind $< 100$ Zellen $\Rightarrow$ Abbruch. Aussage korrekt.
\end{itemize}

% Aufgabe 34c
\part Angenommen das Problem wäre entscheidbar, dann könnte man jede TM so umbauen das sie, beim Wechsel in den Endzustand ein "a" auf das Band schreibt.
Mit einem solchen Umbau könnte man dann das Halteproblem entscheiden. Die TM hält mit einer Eingabe w an  $\Leftrightarrow $ Die TM schreibt ein "a" auf das Band.
Da das Halteproblem allerdings nicht entscheidbar ist, für diese Annahme zu einem Wiederspruch, also ist die Eigenschaft nicht entscheidbar.

\end{parts}

%
% Aufgabe 35
%
\titledquestion {Eine nicht entscheidbare Sprache}
\begin{parts}

% Aufgabe 35a
\part  Es ist zuzeigen das $\overline{H_0} \le P$: $W_n$ sei die Kodierung einer TM die auf dem leeren Band hält. $w$ sei die Eingabe für $\overline{H_0}$. Dann ist die Vorverarbeitungsfunktion $f(w) = W_n \# w$. \\
Also gilt $\overline{H_0} \le P$ und damit ist P nicht semi-entscheidbar.


% Aufgabe 35b
\part Es ist zuzeigen das $\overline{H_0} \le \overline{P}$: $W_\infty$ sei die Kodierung einer TM die nicht auf dem leeren Band hält. $w$ sei die Eingabe für $\overline{H_0}$. Dann ist die Vorverarbeitungsfunktion $f(w) = w\#W_\infty $. \\
Also gilt $\overline{H_0} \le P$ und damit ist $\overline{P}$ nicht semi-entscheidbar.
\end{parts}

%
% Aufgabe 36
%
\titledquestion {Die Landau-Notation}
\begin{parts}

% Aufgabe 36a
\part  Es sind ein $ N$ und ein $C$ gesucht für die gilt $\forall x > N : \mid f(x)\mid \le C  \mid g(x) \mid $. \\ Versuch mit $N = 1$:
\begin{align}
\mid f(x) \mid &= \mid \sqrt{\mid x\mid} + 10 \mid = \sqrt{x} + 10 \mid\text{  Wegen $N =1$ und $x > N$} \notag \\
&= x^{\frac{1}{2}}  \le C * x^3 \notag \\
&\Leftrightarrow \frac{x^{\frac{1}{2}}}{x^3} + \frac{10}{x^3} \le C \notag \\
&\Leftrightarrow \frac{1}{\sqrt{x^{5}}} + \frac{10}{x^3} \le C \notag \\
&\Leftrightarrow 11 \le C \mid\text{ Maximum der linken Seite ist bei $x = 1$}\notag
\end{align}
Für $x > 1$ und $C \ge 11$ gilt die Ungleichung und damit ist $ f\in O(g)$.

\newpage

% Aufgabe 36b
\part Da $f(x)$ und $h(x)$ beide in $O(g)$ liegen muss es zwei N  und C geben für die gilt: \\
$\forall x > N_f : \mid f(x) \mid \le C_f \mid g(x) \mid$ \\
$\forall x > N_h : \mid f(x) \mid \le C_h \mid g(x) \mid$ \\
Falls $\varphi \in O(g)$ muss folgendes gelten:
\begin{align}
\forall x > N \text{: } \mid \varphi(x) \mid  &=\notag \\ 
\mid f(x) + h(x) \mid &\le \mid f(x) \mid + \mid h(x) \mid \notag \\ 
&\le C_f * \mid g(x) \mid + C_h * \mid g(x) \mid \notag \\ 
&\le (C_f + C_h) * \mid g(x) \mid \notag \\ 
&= C * \mid g(x) \mid \text{ mit $ C = C_f + C_h$}\notag
\end{align}
Das $N$ für $\varphi (x)$ ist $N = \text{max}(N_f, N_h)$

% Aufgabe 36c
\part Falls: $f \in O(g)  \Rightarrow  \mid f(x) \mid \le C * \mid g(x) \mid$ also
\begin{align}
\forall x > N  \text{:  } &3^x \le C * 2^x \notag \\ 
&\Leftrightarrow \frac{3^x}{2^x} \le C  \notag \\ 
&\Leftrightarrow \left (\frac{3}{2}\right )^x \le C  \notag 
\end{align}
Da $ \frac{3}{2} > 1$ ist das Wachstum nicht beschränkt und deshalb gilt: $f \notin O(g)$

% Aufgabe 36d
\part Sei $x \ge 1$, dann gilt für alle $i \in \{ 1, \dots, n-1 \} : x^i \le x^n$
\begin{align}
\mid f(x)\mid &\text{ =}  \mid \sum\limits_{i=0}^{n} k_i x^i \mid \text{ } \le\text{ } \sum\limits_{i=0}^{n} \mid k_i \mid * \mid x^i \mid \notag \\
&\text{=} \sum\limits_{i=0}^{n} \mid k_i\mid  * x^i    \le \left( \sum\limits_{i=0}^{n}  \mid k_i \mid \right) * x^n          \notag \\
C &= \left( \sum\limits_{i=0}^{n}  \mid k_i \mid \right) \notag
\end{align}
Da es ein N und C gibt gilt $f \in O(x^n)$

\end{parts}

%
% Aufgabe 37
%
\titledquestion {Unäres vs. logarithmisches Kostenmaß}
\begin{parts}

% Aufgabe 37a
\part  unäres Kostenmaß : $O(n * b) = O(n * \log_2(n))$

% Aufgabe 37b
\part  logarithmisches Kostenmaß : $O(n * b) = O(2^b * b)$
\end{parts}

%
% Aufgabe 38
%
\titledquestion {Rechenschritte einer Turing-Maschine }
\begin{parts}

% Aufgabe 38a
\part Die TM interpretiert die Eingabe als natürliche Zahl in Binärdarstellung und dekrementiert sie so lange, bis 0 auf dem Band steht.

% Aufgabe 38b
\part Der worst-case triff für Eingaben auf, die nur aus Einsen bestehen. Für diesen Fall müssen ingesamt $2^n -1$ Drekementierungen durchgeführt werden. Jede Dekrementation besteht aus maximal $2n$-Schritte (Wenn die Zahl die Form 100000... hat). Zusätzlichen wir die Zahl am Anfang und am Ende einmal komplett durchlaufen.
Zusammen ergibt sich eine Gesamtlaufzeit von:\\ $O(2n + (2^n -1)(2n)) = O(2n + 2n*2^n - 2n) = O(2n *2^n) = O(n 2^n)$
\end{parts}

%
% Aufgabe 39
%
\titledquestion {Polynomielle Laufzeiten}
\begin{parts}

% Aufgabe 39a
\part  Sei $P_B : \mathbb{N}_0 \Rightarrow \mathbb{N}_0$ das Polynom, welches die Laufzeit von B beschränkt und $p_f$ das entsprechende Polynom für f. Sei w ein Wort der Länge n. Wir können die Länge von $f(w)$ mittels $p_f(n)$ abschätzen, da eine TM in polynomische Zeit höchsten polynomial viele Zeichen auf das Band schreiben kann. Ingesamt lässt sich A in polynomialer Zeit entscheiden und z war mit maximal $p_f(n) + p_B(p_f(n))$ Schritten.

% Aufgabe 39b
\part Ansatz analog zum Teil a), nur dass noch die Vorverwarbeitung von C dazu kommt, also ingesamt $p_f(n) + p_g(p_f(n)) + p_C(p_g(p_f(n)))$ Schritte.

% Aufgabe 39c
\part Mit der Vorverarbeitungsfunktion f wie folgt möglich:
\begin{itemize} 
\item Man wählt zunächst 2 Wörter: $w_0 \notin B$ und $w_1 \in B$. Möglich da $B \not= \emptyset \wedge B \not= \sum{*}$
\item Dann ist $f(w) = \begin{cases} w_0 , w \notin A \\ w_1, w\in A \end{cases}$
\end{itemize}
\end{parts}

%
% Aufgabe 40
%
\titledquestion {Kurze Fragen I (Entscheidbarkeit und Berechenbarkeit)}
\begin{parts}

% Aufgabe 40a
\part Falsch, da jede entscheidbare Sprache auch semi-entscheidbar ist.

% Aufgabe 40b
\part Falsch, die charakterischte Funktion für die Sprache $\sum{*}$ ist $\chi_{\sum{*}} = 1$, welche offensichtlich berechenbar ist.

% Aufgabe 40c
\part Falsch, wegen dem Satz von Rice.

% Aufgabe 40d
\part  Richtig, eine TM die die Identitätsfunktion berechnet, kann einfach direkt terminieren.

% Aufgabe 40e
\part  Falsch. Beweis durch Gegenbeispiel: $Y = \emptyset$ und $X \not= \emptyset$

% Aufgabe 40f
\part Richtig, da jede det TM als eine nicht det TM aufgefasst werden kann und es für jede nicht det TM eine äquivalente det TM gibt.
\end{parts}

%
% Aufgabe 41
%
\titledquestion {Weitere kurze Fragen}
\begin{parts}

% Aufgabe 41a
\part  Falsch, wäre H NP-vollständig also $H \in NP$ müsste H entscheidbar sein.

% Aufgabe 41b
\part Falsch, da $P \subseteq NP$ gitl, würde aus der Aussage sofort $P =NP$ folgen, was allerdings (noch) nicht bekannt ist.

% Aufgabe 41c
\part Richtig, man kann einen Faktor $t \in \mathbb{N}$ raten und dann in polynomialer Zeit überprüfuen ob $2 \le t \ln k$ und ob n durch t teilbar ist.

% Aufgabe 41d
\part Richtig. Wenn man eine Aussagelogischeformel f in KNF negiert, dann ist $\neg f$ in DNF. Es gilt: 
\begin{align}
\text{F hat eine erfüllende Belegung } &\Leftrightarrow \neg f \text{ ist nicht die Tautologie} \notag \\
f \in \text{KNF-SAT} &\Leftrightarrow \neg f \notin \text{DNF-TAU}
\end{align}
$\Rightarrow $ Wäre DNF-TAU in polynomialer Zeit entscheidbar, dann könnte man auch KNF-SAT in polynomialer Zeit entscheiden und damit auch alle anderen Probleme in NP.
Also wäre P = NP, was allerdings (noch) nicht bewiesen ist.
\end{parts}

%
% Aufgabe 42
%
\titledquestion {CLIQUE $\le_p$ INDEPENDENT - SET}
\begin{parts}

% Aufgabe 42a
\part  Sei $G =(V,E)$ ein ungerichteter Graph und k eine Zahl. Wir definieren $f(G,k) = (G', k)$ mit $G'(V,E')$ und $E' = \{(v_1,v_2) \in V \times V \mid v_1 \not= v_2 \wedge (v_1,v_2) \notin E\}$, also die Menge aller Kanten die nicht in E liegen. Diese Funktion f ist total und in polynomialzeit berechenbar.
\begin{align}
(G,k) \in \text{CLIQUE} \Leftrightarrow f(G,k) \in \text{INDEPENDET - SET}
\end{align}
 und damit ist INDEPENDET - SET NP-hart.

% Aufgabe 42b
\part Es ist möglich eine Menge M mit k Knoten zuraten und dann zu überprüfen ob $M \in IS$ ist. Es gibt maximal $\frac{n(n-1)}{2}$ Kanten, also ist das testen in polynomialer Zeit möglich. Zusammen mit dem Ergebnis aus (a) bedeutet das das INDEPENDENT - SET NP-vollständig ist
\end{parts}

%
% Aufgabe 43
%
\titledquestion {VERTEX - COVER}
\begin{parts}

% Aufgabe 43a
\part  Sei $G = (V,E) $ ein ungerichteter Graph mit $n = \mid V\mid$ und $k \in \mathbb{N}$. Wir definieren $f(G,k) = (G,n-k)$. Diese Funktion f ist total und in polynomialzeit berechenbar.\\
Wir müssen zeigen: $(G,k) \in \text{ INDEPENDENT - SET} \Leftrightarrow f(G,k) \in \text{VERTEX - COVER}$\\
"$\Rightarrow$": Sei $(G,k) \in IS$ und sei S ein IS der Größe k. Wir behaupten, dass $ V \backslash S$ ein VC der Größe $n-k$ ist, d. h. dass für alle Kanten $e \in E$ min ein Knoten der Kante in $V \backslash S$ liegt.\\ Angenommen es gäbe eine Kante $e' = \{v_1,v_2\} \in E $ mit $v_1 \notin V \backslash S$ und $v_2 \notin V \backslash S$, d.h. $v_1,v_2 \in S$. Eine solche Kante gibt es nicht, da $S \in IS$ ist.
\\"$\Leftarrow$": Sei $(G,n-k) \in VC$ und C ein VC der Größe $n-k$. Dann ist $V \backslash C $ ein IS der Größe k, d. h. es gibt keine Kannten zwischen ja 2 Knoten aus $V \backslash C$. Angenommen es gäbe eine Kanten mit $ e = {v_1,v_2} \in E$ mit $v_1,v_2 \in V \backslash C$, d.h. $v_1,v_2 \notin C$. Eine solche Kanten gibt es nicht, da C eine Knotenüberdeckung ist.\\
Daraus folgt das VC NP-hart ist.

% Aufgabe 43b
\part Es ist möglich eine Menge M mit k Knoten zuraten und dann zu überprüfen ob $M \in VC$ ist. Es gibt maximal $\frac{n(n-1)}{2}$ Kanten, also ist das testen in polynomialer Zeit möglich. Zusammen mit dem Ergebnis aus (a) bedeutet das das VERTEX - COVER NP-vollständig ist
\end{parts}

\end{questions}
\end{document}