\documentclass[a4]{exam}
\usepackage[utf8]{inputenc}
\usepackage{amsmath}
\usepackage{amssymb}
\usepackage{algorithm}
\usepackage{algorithmicx}
\usepackage{algpseudocode}

\renewcommand{\author}{Michael Krane}
\newcommand{\gruppe}{Gruppe E2}
\newcommand{\assignment}{Übungsblatt 2}
\newcommand{\class}{BeKo 2013/14}

\pagestyle{headandfoot}
\firstpageheadrule
\runningheadrule
\header{\author}{\gruppe : \assignment}{\class}
\firstpagefooter{}{}{Seite \thepage\ von \numpages}
\runningfooter{}{}{Seite \thepage\ von \numpages}

\qformat{\underline{\textbf{Aufgabe \thequestion}} \thequestiontitle\hfill }

\begin{document}
\begin{questions}
\setcounter{question}{4}

%
% Aufgabe 5
%
\titledquestion {Zählen von Turingmaschinen}
\begin{parts}

% Aufgabe 5a
\part Für die deterministische TM gibt es ingesamt 18 verschiedene "rechte Seiten" der Übergangsfunktion ($|\{z_1,z_2,z_3\}| * |\{a,\Box\}|* |\{R,L,N\}| = 18$) und 6 verschiedene "linke Seiten" der Übergangsfunktion ($|\{z_1,z_2,z_3\}| * |\{a,\Box\}|= 6$) , Also gibt es ingesamt $18^6 = 34012224$ mögliche Übergangsfunktionen.

% Aufgabe 5b
\part Für die nichtdeterministische TM gibt es ingesamt $2^{18}$ verschiedene "rechte Seiten" der Übergangsfunktion ($|\mathcal{P}(\{z_a,z_b\}\times |\{x,y,\Box\} \times\{R,L,N\})| = 2^{18}$) und 6 verschiedene "linke Seiten" der Übergangsfunktion  ($|\{z_a,z_b\}| * |\{x,y,\Box\}|= 6$). Es gibt also ingesamt ${2^{18}}^6 = 262144^{6} = 3,245 * 10^{32}$ mögliche Übergangsfunktionen.

% Aufgabe 5c
\part Die Überlegung aus (a) kann in eine verallgemeinerte Formel umgewandelt werden: Die Anzahl der Übergangsfunktionen kann mit 
$ (3 * m * n)^{m*n} $  berechnet werden, wobei $ m = |\Gamma| $ und $ n =  |Z|$.

% Aufgabe 5d
\part Die Überlegung aus (b) kann auch wieder in eine verallgemeinerte Formel umgewandelt werden: Die Anzahl der Übergangsfunktionen kann mit:
$2^{(3mn)^{m*n}} = 2^{3m^2n^2}$ berechnet werden, wobei $ m = |\Gamma| $ und $ n =  |Z|$.
\end{parts}

%
% Aufgabe 6
%
\titledquestion{Was macht diese Turingmaschine?}
\begin{parts}

% Aufgabe 6a
\part Für die Eingabe 'ab' durchläuft die TM folgende Schritte: \\
$z_0 ab \vdash 
az_0b \vdash 
abz_0\Box \vdash 
a z_sb \#  \vdash 
a \#z_b\#  \vdash
a \#\#z_b\Box \vdash
a \#z_r\#b \vdash
a z_s\#\#b \vdash
z_sa\#\#b \vdash
\#z_a\#\#b  \vdash \\
\#\#z_a\#b \vdash 
\#\#\#z_ab \vdash
\#\#\#bz_a\Box \vdash
\#\#\#z_rba \vdash
\#\#z_r\#ba \vdash
\#z_s\#\#ba \vdash
z_s\#\#\#ba\vdash
z_s\Box \#\#\#ba \vdash
z_d\#\#\#ba \vdash
z_d\#\#ba \vdash
z_d\#ba \vdash
z_dba \vdash
z_Eba $ \\

Für die Eingabe 'ab' : $z_0ab \vdash^* z_Eba$\\
Für die Eingabe 'aabb': $z_0aabb \vdash^* z_Ebbaa$.

% Aufgabe 6b
\part Die TM schreibt die Eingabe rückwärts auf das Band und löscht danach die ursprüngliche Eingabe.

% Aufgabe 6c
\part $z_0$: Es wird das rechte Ende der Eingabe mit einem \#-Zeichen makiert. Danach wechselt die TM nach $z_s$.\\
$z_s$: Es wird vom rechten Ende des Wortes aus nach einem a oder b gesucht, welches mit \# überschrieben wird. Bei einem a geht es mit $z_a$ weiter, bei einem b geht es mit $z_b$ weiter. Falls es kein a oder b mehr gibt, wechselt die Turingmaschine nach  $z_d$.\\
$z_a$: Das gefunde a wird an das rechte Wortende geschrieben. Danach wechselt die TM nach $z_r$.\\
$z_b$: Das gefunde b wird an das rechte Wortende geschrieben. Danach wechselt die TM nach $z_r$.\\
$z_r$: Der Lesekopf wird wieder auf das ürsprüngliche Ende der Eingabe gesetzt, dann wechselt die TM nach $z_s$.\\
$z_d$: Der Lesekopf wird auf das erste Zeichen des umgedrehten Wortes gesetzt und löscht dabei alle erzeugen \#.  Danach TM wechselt in den Endzustand $z_e$.\\
$z_E$: Die TM befindet sich im Endzustand und terminiert.

\end{parts}

%
% Aufgabe 7
%
\titledquestion{Turingmaschine für eine reguläre Sprache}
\begin{parts}

% Aufgabe 7a
\part $G = (V, \Sigma, P, S)$ \\
$V = \{S,A, B\}$ \\
$\Sigma = \{a,b\}$\\
$P = \{ S \rightarrow bS \mid aA , A \rightarrow aA \mid ab\}$ \\

% Aufgabe 7b
\part Man kann $aa^*a$ in $aaa^*$ umformen. Damit ist $ L: b^* aaa^* b $. \\\\$
\delta(z_0,a) = (z_1, \Box, R) $ $ z_0$ erkennt $b^*$ und wechselt bei einem a nach $z_1$ .\\ $
\delta(z_0,b) = (z_0, \Box, R) \\
\delta(z_0,1) = (z_4, \Box, R) \\
\delta(z_0,\Box) = (z_E, \Box, N) \\\\
\delta(z_1,a) = (z_2, \Box, R) $  $z_1$ erkennt das zweite a und wechselt nach  $z_2$. Ansonsten in den\\$ 
\delta(z_1,b) = (z_4, \Box, R) $  Fehlerzustand $z_4$. \\ $
\delta(z_1,1) = (z_4, \Box, R) \\
\delta(z_1,\Box) = (z_E, \Box, N) \\\\
\delta(z_2,a) = (z_2, \Box, R)$ $z_2$ erkennt $a^*$, wechselt bei einem b nach $z_3$ und sonst in den Fehlerzustand $z_4$. \\ $ 
\delta(z_2,b) = (z_3, \Box, R) \\
\delta(z_2,1) = (z_4, \Box, R) \\
\delta(z_2,\Box) = (z_E, \Box, N) \\\\ 
\delta(z_3,a) = (z_4, \Box, R) $ Falls das Band leer ist, wird eine '1' geschrieben und in den Endzustand \\$
\delta(z_3,b) = (z_4, \Box, R) $   gewechselt, ansonsten in den Fehlerzustand $z_4$.\\ $
\delta(z_3,1) = (z_4, \Box, R) \\
\delta(z_3,\Box) = (z_E, 1, N) \\\\ 
\delta(z_4,a) = (z_4, \Box, R) $ Fehlerzustand. Überschreibt Band mit $\Box$ und wechselt in den Endzustand.\\ $
\delta(z_4,b) = (z_4, \Box, R) \\
\delta(z_4,1) = (z_4, \Box, R) \\
\delta(z_4,\Box) = (z_E, \Box, N) \\\\
\delta(z_E,a) = (z_E, a, N)  $ Endzustand. Übergangsfunktion vorgegeben.  \\ $
\delta(z_E,b) = (z_E, b, N) \\
\delta(z_E,1) = (z_E, 1, N) \\
\delta(z_E,\Box) = (z_E, \Box, N) \\ 
$
\end{parts}

%
% Aufgabe 8
%
\titledquestion{Wiederholung: Funktionen Teil II }
\begin{parts}

% Aufgabe 8a
\part $f_a : \mathbb{R} \rightarrow \mathbb{R}, f_a(x) := 2 $ \\ Da  $f_a(1) = f_a(2) $ aber $ 1 \not= 2$ also ist $f_a$ nicht injektiv. \\ Nicht surjektiv da es z.B. für  $y \in \mathbb{R} = 3$  kein $ x \in \mathbb{R}$ mit $ f_b(x) = 3$ gibt.

% Aufgabe 8b
\part $f_b : \mathbb{N} \rightarrow \mathbb{R}, f_b := x $ \\  Aus $ f_b(a) = f_b(b)  $ folgt $ a  = b$ also ist $f_b(x)$ injektiv. \\  Nicht surjektiv da es z.B. für $y \in \mathbb{R} = 2,5$  es  kein $ x \in \mathbb{N}$ mit $ f_b(x) = 2,5$ gibt.

% Aufgabe 8c
\part $f_c : \mathbb{R} \rightarrow \{2\}, f_c(x) := 2 $ \\ Da  $f_c(1) = f_c(2) $ aber $ 1 \not= 2$ also ist $f_c$ nicht injektiv. \\ $f_c$ ist surjektiv, da es für alle $ y \in \{2\}$ ein Urbild in $\mathbb{R}$ gibt.

% Aufgabe 8d
\part $f_d : \mathbb{R} \rightarrow \mathbb{R}, f_d(x) := x $ \\ Aus $ f_d(a) = f_d(b)  $ folgt $ a  = b$ also ist $f_d(x)$ injektiv. \\
Für alle $y \in \mathbb{R} = Y$ existiert ein $x \in \mathbb{R} = X$, falls $ X = Y$ und $f_d(x) = x$ also die Identität  von x ist.
\end{parts}

\end{questions}
\end{document}