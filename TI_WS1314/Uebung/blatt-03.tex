\documentclass[a4]{exam}
\usepackage[utf8]{inputenc}
\usepackage{amsmath}
\usepackage{amssymb}
\usepackage{algorithm}
\usepackage{algorithmicx}
\usepackage{algpseudocode}

\renewcommand{\author}{Michael Krane}
\newcommand{\gruppe}{Gruppe E2}
\newcommand{\assignment}{Übungsblatt 3}
\newcommand{\class}{BeKo 2013/14}

\pagestyle{headandfoot}
\firstpageheadrule
\runningheadrule
\header{\author}{\gruppe : \assignment}{\class}
\firstpagefooter{}{}{Seite \thepage\ von \numpages}
\runningfooter{}{}{Seite \thepage\ von \numpages}

\qformat{\underline{\textbf{Aufgabe \thequestion}} \thequestiontitle\hfill }

\begin{document}
\begin{questions}
\setcounter{question}{8}   %Fragennummer anpassen

%
% Aufgabe 9
%
\titledquestion {Überabzahlbarkeit und Diagonalisierung} 
\begin{parts}

% Aufgabe 9a
\part Wenn $I_1$ abzählbar ist, kann man mit der Funktion $f_n : \mathbb{N}_0 \Rightarrow I_1 $, wobei $f_n(x)$ die x. Dezimalstelle der n-ten Zahl angibt eine Liste erstellen, die alle Zahlen $\in I_1$ enthält. \\
\begin{tabular}{r| r r r r r r r}
n & $f_n(0) $ & $f_n(1) $ &$f_n(2) $ &$f_n(3) $ &$f_n(4) $ & $\dots$ \\ \hline
0& -0, & 1 & 2 & 3 & 4\\
1& 0, & 1 & 2 & 3 & 4\\
2& -0, & 2 & 3 & 4 & 5\\
3& 0, & 2 & 3 & 4 & 5\\
\vdots \\
\end{tabular}

Wenn man nun eine neue Zahl $\in I_1$ mit  $g:\mathbb{N}_0 \Rightarrow  \mathbb{N}_0$ mit $g(n) = \begin{cases} 0 &$falls $ n = 0 \wedge f_0(0) = -0 \\ -0 &$falls $ n = 0 \wedge f_0(0) = 0 \\ 1 & $falls $ f_n(n) = 2 \\    2 & $falls $ f_n(n) \not= 2 \\    \end{cases} $ definieren. Da die neue Zahl, per Definition nicht in der Liste enthalten seien kann, muss $I_1$ überabzählbar groß sein.

% Aufgabe 9b
\part $\varphi_r: I_r \Rightarrow I_1$ , $ \varphi_r(x) = \frac{x}{r} $ mit $\varphi^{-1} : I_1 \Rightarrow I_r$ , $\varphi^{-1}(x) = rx$

% Aufgabe 9c
\part Die Bildmenge der Funktion $f(x) = \tan(x)$ auf dem Intervall $\left]-\frac{\pi}{2}, \frac{\pi}{2}\right[$ ist $\mathbb{R}$. Man kann für jedes $r \in \mathbb{R}$ eine Funktion $f_r : I_r \Rightarrow \mathbb{R} ,f_r(x) = \tan (\frac{x}{r}*\frac{\pi}{2}) $ definieren. $f_r$ ist bijektiv da die Umkehrfunktion $f_r^{-1}: I_r \Rightarrow \mathbb{R}, f_r^{-1}(x) = \frac{2r}{\pi}\arctan(x)$ existiert.

% Aufgabe 9d
\part Es gibt genau dann eine bijektive Funktion $g_n : \mathbb{N}^n \Rightarrow \mathbb{R}$, wenn $\mathbb{N}^n$ und $\mathbb{R}$ gleich mächtig sind. Mit vollständiger Induktion kann bewiesen werden, das alle n-Tupel aus $\mathbb{N}$ gleichmächtig und damit abzählbar sind: Für $\mathbb{N}^2$ gibt es eine Funktion  $f_2 : \mathbb{N} \times \mathbb{N} \Rightarrow \mathbb{N}_0$ , $f_2(n,m) = \left(\displaystyle\sum\limits_{i=0}^{m+n} i\right) +n$ Für $n=3$ gibt es eine rekursive Funktion $f_3 : \mathbb{N} \times \mathbb{N} \times \mathbb{N} \Rightarrow \mathbb{N}_0$, $f_3(m,n,k) = f_2(m,f_2(n,k))$. Für $ n \ge 3$ kann diese Funktion zu $f_2 : \mathbb{N}^n \Rightarrow \mathbb{N}_0$, $f_n(m_1,\dots,m_n) = f_2(m_1,f_{n-1}(m_2,\dots,m_n))$ verallgemeinert werden.
 Da die Menge der reellen Zahlen aber überabzählbar ist, sind $\mathbb{N}^n$ und $\mathbb{R}$ nicht gleichmächtig, daraus folgt das es keine bijektive Funtion $g_n : \mathbb{N}^n \Rightarrow \mathbb{R}$ geben kann.
\end{parts}

%
% Aufgabe 10
%
\titledquestion {Bandbeschränkt vs. Zeitbeschränkt} 
\begin{parts}

% Aufgabe 10a
\part Stimmt nur, wenn alle $b(n)$ besuchtet Stellen links oder rechts von der Startposition liegen und der Kopf niemals eine Bwegung in die andere Richtung macht, d.h. jede Stelle wird nur genau einmal besucht.

% Aufgabe 10b
\part Stimmt. Mit $t(n)$ Schritten kann sich der Lesekopf maximal $t(n)$-mal bewegen. Geht der Kopf immer nur in eine Richtung können maximal $t(n) +1$ Stellen besucht werden. $t(n)$ Stellen rechts/links von der Startposition, plus die Startposition selber. Alle anderen Bewegungenskombinationen haben $ <b(n)$ verschiedene besuchte Stellen, da mindestens eine Stellen doppelt besucht werden muss.

% Aufgabe 10c
\part Stimmt nicht. Da das Band unendlich lang ist, ist es möglich das die TM die ganze Zeit über nur nach links oder rechts läuft und sich deswegen keine Konfiguration wiederholt.

% Aufgabe 10d
\part Stimmt. Da es bei einer bandbeschränkter TM nur endlich viel Platz gibt und das Arbeitsalphabet eine endliche Menge ist, gibt es auch auch nur endlich viele Permutationen für die Konfiguration einer  bandbeschränkten TM. Da aber unendlich viele Schritte gemacht werden, muss mindestens eine Konfiguration mehrmals erreicht werden.

% Aufgabe 10e
\part Stimmt. Wenn eine Konfiguration vor erreichen des Endzustandens mehrfach erreicht wird, bebeutet dass das die TM in einer Endlosschleife und kann nicht terminieren. Sie führt also unendlich viele Schritte aus.
\end{parts}

\newpage
%
% Aufgabe 11
%
\titledquestion {Zweiband-Turingmaschine} 
Sei $ M = (\{z_1,z_E\},\{a,b\},\{a,b,\Box\},\delta,z_1,\Box,\{z_E\})$  eine Mehrband-Turingmaschine mit \\$\delta : Z \times \Gamma^k \Rightarrow Z \times  \Gamma^k \times \{L,R,N\}^k$ und $ k = 2$ Bändern. \\\\ 
 In $z_1$ wird die Eingabe von Band 1 von links nach rechts gelesen und nacheinander mit $\Box$ überschrieben. Gleichzeitig wird das gelesene Zeichen von nach rechts nach links auf das Ausgabeband geschrieben. Dadurch wird die Eingabe umgedreht. Sobald die Eingabe komplett abgearbeitet ist wechselt die TM in den Endzustand $z_E$\\$
\delta(z_1,(a,a)) = (z_1,(a,a),(N,N)) \\
\delta(z_1,(a,b)) = (z_1,(a,b),(N,N))\\
\delta(z_1,(a,\Box)) = (z_1,(\Box,a),(R,L))\\ 
\delta(z_1,(b,a)) = (z_1,(b,a),(N,N))\\
\delta(z_1,(b,b)) = (z_1,(b,b),(N,N))\\
\delta(z_1,(b,\Box)) = (z_1,(\Box,b),(R,L))\\ 
\delta(z_1,(\Box,a)) = (z_1, (\Box,a), (N,N))\\
\delta(z_1,(\Box,b)) = (z_1, (\Box,b), (N,N))\\
\delta(z_1,(\Box,\Box)) = (z_E, (\Box,\Box ), (N,N))$\\\\
$z_E$ ist der Endzustand. Hier passiert nichts. \\$
\delta(z_E,(a,a)) = (z_E,(a,a),(N,N)) \\
\delta(z_E,(a,b)) = (z_E,(a,b),(N,N))\\
\delta(z_E,(a,\Box)) = (z_E,(a\Box),(N,N))\\ 
\delta(z_E,(b,a)) = (z_E,(b,a),(N,N))\\
\delta(z_E,(b,b)) = (z_E,(b,b),(N,N))\\
\delta(z_E,(b,\Box)) = (z_E,(\Box,b),(N,N))\\ 
\delta(z_E,(\Box,a)) = (z_E, (\Box,a), (N,N))\\
\delta(z_E,(\Box,b)) = (z_E, (\Box,b), (N,N))\\
\delta(z_E,(\Box,\Box)) = (z_E, (\Box,\Box), (N,N))
$
\end{questions}
\end{document}