\documentclass[a4]{exam}
\usepackage[utf8]{inputenc}
\usepackage{amsmath}
\usepackage{amssymb}
\usepackage{algorithm}
\usepackage{algorithmicx}
\usepackage{algpseudocode}

\renewcommand{\author}{Michael Krane}
\newcommand{\gruppe}{Gruppe E2}
\newcommand{\assignment}{Übungsblatt 6}
\newcommand{\class}{BeKo 2013/14}

\pagestyle{headandfoot}
\firstpageheadrule
\runningheadrule
\header{\author}{\gruppe : \assignment}{\class}
\firstpagefooter{}{}{Seite \thepage\ von \numpages}
\runningfooter{}{}{Seite \thepage\ von \numpages}

\qformat{\underline{\textbf{Aufgabe \thequestion}} \thequestiontitle\hfill }

\begin{document}
\begin{questions}
\setcounter{question}{19} 

%
% Aufgabe 20
%
\titledquestion {Totale Funktionen} 
Laut Aufgabe ist die Menge der gültigen Programme $L$ rekursiv aufzählbar. Das bedeutet das es eine surjektive Funktion $F : \mathbb{N}_0 \Rightarrow L$ gibt, die einer Zahl n das n-te Programm in der Aufzählung zuordnet. Wenn man nun eine neues Programm definieren, mit $h: \mathbb{N}_0 \Rightarrow \mathbb{N}_0$, $h(x) = F(x) + 1$, welches die Ausgabe des x-ten Programmes + 1 berechnet. Diese Funktion ist laut Definition berechenbar und total, muss also in der  Menge der gültigen Programme L sein. Sei jetzt i der Index von dem Programm $h(x)$ in der Aufzählung $L$. Dann würde $F(i) = h(i) = F(i) +1$ gelten. Das führt allerdings zu einem Wiederspruch und daher kann $h(x)$ nicht Teil der Aufzählung sein. Das wiederspricht der 2. Eigenschaft des Sprache (jede total Funktion kann mit einem Programm der Sprache beschrieben werden), weshalb die Sprache so nicht existieren kann.
%
% Aufgabe 21
%
\titledquestion {primitiv-rekursive Funktionen}
 $\chi_{\{0\}} : \mathbb{N}_0 \Rightarrow \mathbb{N}_0$, $f(n) = \begin{cases} 1 &$, $ x = 0  \\ 0 &$, $x > 0 \\  \end{cases}$  siehe Blatt 5 Aufgabe 18.
\begin{parts}
% Aufgabe 21a
\part  
$leq(x,y) = \chi_{\{0\}}(x-y)$

% Aufgabe 21b
\part 
$geq(x,y) = \chi_{\{0\}}(y-x)$

% Aufgabe 21c
\part
$eq(x,y) = mult(leq(x,y), geq (x,y))$

\end{parts}

%
% Aufgabe 22
%
\titledquestion {$\mu$-rekursive Funktionen}
\begin{parts}

% Aufgabe 22a
\part  Mit dem $\mu$-Operator ist $\mu f_1 : \mathbb{N}^1_0 \Rightarrow \mathbb{N}_0$, $\mu f_1(y) = min \{n \mid f_1(n,y) = 0\}$, mit $f_1(x,y) = y-x$. $f_1$ ist genau dann = 0, wenn $ y-n = 0 \Leftrightarrow y = n$ ist. \\
Also ist $\mu f_1 : \mathbb{N}^1_0 \Rightarrow \mathbb{N}_0$, $\mu f_1(x) = x$, also die Identitätsfunktion.

% Aufgabe 22b
\part 
\begin{enumerate}
\item $\sqrt[x]{y} = n \Leftrightarrow y \leq n^x \Leftrightarrow y - n^x \leq 0$. Die Funktion bricht also ab, sobald $n^x \geq y$ ist.\\ $n^x$ ist primitiv-rekursiv, siehe Blatt 5 Aufgabe 18(c).\\
$\mu f_{pot} : \mathbb{N}^2_0 \Rightarrow \mathbb{N}_0$, $\mu f_{pot}(x,y) = min \{n \mid f_{pot}(n,x,y) = 0\}$ \\
$f_{pot} : \mathbb{N}^3_0 \Rightarrow \mathbb{N}_0$, $f_{pot}(n,x,y) = y - n^x$\\



\item Für $n = x$ ist $1-eq(x,n) = 0$ und $1-eq(y,n) =1$, also $(1-eq(x,n)) *(1-eq(y,n)) = 0$. Für $n = y$ analog. Da n hochgezählt wird tritt wird $f_{min}(n,x,y)$ zuerst bei dem kleineren Parameter $=0$.\\
$\mu f_{min} : \mathbb{N}^2_0 \Rightarrow \mathbb{N}_0$, $\mu f_{min}(x,y) = min \{n \mid f_{min}(n,x,y)= 0\}$ \\
$f_{min} : \mathbb{N}^3_0 \Rightarrow \mathbb{N}_0$, $f_{min}(n,x,y) = (1-eq(x,n)) * (1-eq(y,n))$ 
\end{enumerate}

\end{parts}
\end{questions}
\end{document}