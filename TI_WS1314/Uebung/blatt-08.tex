\documentclass[a4]{exam}
\usepackage[utf8]{inputenc}
\usepackage{amsmath}
\usepackage{amssymb}
\usepackage{algorithm}
\usepackage{algorithmicx}
\usepackage{algpseudocode}

\renewcommand{\author}{Michael Krane}
\newcommand{\gruppe}{Gruppe E2}
\newcommand{\assignment}{Übungsblatt 8}
\newcommand{\class}{BeKo 2013/14}

\pagestyle{headandfoot}
\firstpageheadrule
\runningheadrule
\header{\author}{\gruppe : \assignment}{\class}
\firstpagefooter{}{}{Seite \thepage\ von \numpages}
\runningfooter{}{}{Seite \thepage\ von \numpages}

\qformat{\underline{\textbf{Aufgabe \thequestion}} \thequestiontitle\hfill }

\begin{document}
\begin{questions}
\setcounter{question}{25}

%
% Aufgabe 26
%
\titledquestion {Franz der Frisör (Diagonalisierung reloaded)}


%
% Aufgabe 27
%
\titledquestion {Reduktionen}


%
% Aufgabe 28
%
\titledquestion {Ein neues Halteproblem}


%
% Aufgabe 29
%
\titledquestion {Abgeschlossenheit bezüglich Reduktionen}
\begin{parts}

% Aufgabe 29a
\part  

% Aufgabe 29b
\part 

\end{parts}

\end{questions}

\end{document}