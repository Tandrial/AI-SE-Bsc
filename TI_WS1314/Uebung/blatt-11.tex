\documentclass[a4]{exam}
\usepackage[utf8]{inputenc}
\usepackage{amsmath}
\usepackage{amssymb}
\usepackage{algorithm}
\usepackage{algorithmicx}
\usepackage{algpseudocode}

\renewcommand{\author}{Michael Krane}
\newcommand{\gruppe}{Gruppe E2}
\newcommand{\assignment}{Übungsblatt 10}
\newcommand{\class}{BeKo 2013/14}

\pagestyle{headandfoot}
\firstpageheadrule
\runningheadrule
\header{\author}{\gruppe : \assignment}{\class}
\firstpagefooter{}{}{Seite \thepage\ von \numpages}
\runningfooter{}{}{Seite \thepage\ von \numpages}

\qformat{\underline{\textbf{Aufgabe \thequestion}} \thequestiontitle\hfill }

\begin{document}
\begin{questions}
\setcounter{question}{37}

%
% Aufgabe 38
%
\titledquestion {Rechenschritte einer Turing-Maschine }
\begin{parts}

% Aufgabe 38a
\part Die TM interpretiert die Eingabe als natürliche Zahl in Binärdarstellung und dekrementiert sie so lange, bis 0 auf dem Band steht.

% Aufgabe 38b
\part Der worst-case triff für Eingaben auf, die nur aus Einsen bestehen. Für diesen Fall müssen ingesamt $2^n -1$ Drekementierungen durchgeführt werden. Jede Dekrementation besteht aus maximal $2n$-Schritte (Wenn die Zahl die Form 100000... hat). Zusätzlichen wir die Zahl am Anfang und am Ende einmal komplett durchlaufen.
Zusammen ergibt sich eine Gesamtlaufzeit von:\\ $O(2n + (2^n -1)(2n)) = O(2n + 2n*2^n - 2n) = O(2n *2^n) = O(n 2^n)$
\end{parts}

%
% Aufgabe 39
%
\titledquestion {Polynomielle Laufzeiten}
\begin{parts}

% Aufgabe 39a
\part  Sei $P_B : \mathbb{N}_0 \Rightarrow \mathbb{N}_0$ das Polynom, welches die Laufzeit von B beschränkt und $p_f$ das entsprechende Polynom für f. Sei w ein Wort der Länge n. Wir können die Länge von $f(w)$ mittels $p_f(n)$ abschätzen, da eine TM in polynomische Zeit höchsten polynomial viele Zeichen auf das Band schreiben kann. Ingesamt lässt sich A in polynomialer Zeit entscheiden und z war mit maximal $p_f(n) + p_B(p_f(n))$ Schritten.

% Aufgabe 39b
\part Ansatz analog zum Teil a), nur dass noch die Vorverwarbeitung von C dazu kommt, also ingesamt $p_f(n) + p_g(p_f(n)) + p_C(p_g(p_f(n)))$ Schritte.

% Aufgabe 39c
\part Mit der Vorverarbeitungsfunktion f wie folgt möglich:
\begin{itemize} 
\item Man wählt zunächst 2 Wörter: $w_0 \notin B$ und $w_1 \in B$. Möglich da $B \not= \emptyset \wedge B \not= \sum{*}$
\item Dann ist $f(w) = \begin{cases} w_0 , w \notin A \\ w_1, w\in A \end{cases}$
\end{itemize}
\end{parts}

%
% Aufgabe 40
%
\titledquestion {Kurze Fragen I (Entscheidbarkeit und Berechenbarkeit)}
\begin{parts}

% Aufgabe 40a
\part Falsch, da jede entscheidbare Sprache auch semi-entscheidbar ist.

% Aufgabe 40b
\part Falsch, die charakterischte Funktion für die Sprache $\sum{*}$ ist $\chi_{\sum{*}} = 1$, welche offensichtlich berechenbar ist.

% Aufgabe 40c
\part Falsch, wegen dem Satz von Rice.

% Aufgabe 40d
\part  Richtig, eine TM die die Identitätsfunktion berechnet, kann einfach direkt terminieren.

% Aufgabe 40e
\part  Falsch. Beweis durch Gegenbeispiel: $Y = \emptyset$ und $X \not= \emptyset$

% Aufgabe 40f
\part Richtig, da jede det TM als eine nicht det TM aufgefasst werden kann und es für jede nicht det TM eine äquivalente det TM gibt.
\end{parts}

\end{questions}
\end{document}