\documentclass[a4]{exam}
\usepackage[utf8]{inputenc}
\usepackage{amsmath}
\usepackage{amssymb}
\usepackage{algorithm}
\usepackage{algorithmicx}
\usepackage{algpseudocode}

\renewcommand{\author}{Michael Krane}
\newcommand{\gruppe}{Gruppe E2}
\newcommand{\assignment}{Übungsblatt 1}
\newcommand{\class}{BeKo 2013/14}

\pagestyle{headandfoot}
\firstpageheadrule
\runningheadrule
\header{\author}{\gruppe : \assignment}{\class}
\firstpagefooter{}{}{Seite \thepage\ von \numpages}
\runningfooter{}{}{Seite \thepage\ von \numpages}

\qformat{\underline{\textbf{Aufgabe \thequestion}} \thequestiontitle\hfill }

\begin{document}
\begin{questions}

%
% Aufgabe 1
%

\titledquestion {Chomsky-Hierachie}
\begin{parts}

% Aufgabe 1a
\part Zunächst gilt Typ-3 $\subseteq$ Typ-2 $\subseteq$ Typ-1 $\subseteq$ Typ-0 \\\\
Typ-0 Grammatiken: Eine Grammatik G ist vom Typ-0, wenn keine Einschränkungen vorliegen. \\ \\
Typ-1 Grammatiken: Eine Grammatik G heißt kontextsensitiv (Typ-1), wenn alle Produktions-\\regeln die folgende Form haben: $\alpha_1 A \alpha_2 \rightarrow \alpha_1 \beta \alpha_2 $ mit $ A \in V, \alpha_1, \alpha_2 \in (V  \cup \Sigma)^* , \beta \in (V  \cup \Sigma)^+$ \\ \\
Typ-2 Grammatiken: Eine Grammatik G heißt kontextfrei (Typ-2), wenn alle Produktionsregeln folgende Form haben: $A \rightarrow \beta $ mit $ A \in V, \beta \in (V  \cup \Sigma)^+$\\ \\
Typ-3 Grammatiken: Eine Grammatik G heißt regulär (Typ-3), wenn alle Produktionsregeln folgende Form haben: $A \rightarrow aB $ oder $A \rightarrow a$  mit $ A,B \in V, a \in \Sigma$ \\ 

% Aufgabe 1b
\part \begin{itemize}
\item $ L_1 $ ist eine Typ-2 Grammatik, da es mit Typ-3 nicht möglich ist, mehr als ein Nichtterminalsymbol pro Ableitungsschritt hinzuzufügen. Dies ist allerdings nötig, damit die Anzahl der as und bs gleichbliebt.
\item $L_2$ ist eine Typ-3 Grammatik, da die Anzahl der as und bs nicht voneinander abhängt. Deswegen reicht es, wenn man pro Ableitungsschritt ein Terminalsymbol hinzufügt wird.
\item $L_3$ ist eine Typ-1 Grammatik, da es mindestens eine Produktionsregel mit der Form  $\alpha_1 A \alpha_2 \rightarrow \alpha_1 \beta \alpha_2 $ geben muss,  um die gleiche Anzahl von terminal Symbolen an unterschiedlichen Stellen zu erzeugen.\\
\end{itemize}

% Aufgabe 1c
\part $G_1 = (V, \Sigma, P, S)$ \\
$V = \{S,A,B\}$ \\
$\Sigma = \{a,b\}$\\
$P = \{ S \rightarrow aAb, A \rightarrow aAb \mid B, B \rightarrow bBa \mid ba\}$ \\
Da alle Produktionsregeln die Form $A \rightarrow \beta $ mit $ A \in V, \beta \in (V  \cup \Sigma)^+$ haben, ist  die Grammatik vom Typ-2.\\
\end{parts}

%
% Aufgabe 2
%
\titledquestion{Wiederholung: Chomsky-Grammatiken}
\begin{parts}

% Aufgabe 2a
\part $ L_1 = \{ a^{n}b^m \mid n,m \in \mathbb{N} , n >1 \} $

% Aufgabe 2b
\part $ L_2 = \{ a^{2n}b^{3n} \mid n \in \mathbb{N}  \} $

% Aufgabe 2c
\part aabbb $ \in L_1 \cap L_2 $ 

% Aufgabe 2d
\part $ L_1 \cup L_2 = \{ a^{2n}b^{3n} \mid n \in \mathbb{N}  \} $ da $ L_2 \subseteq  L_1$\\
\end{parts}

%
% Aufgabe 3
%
\titledquestion{Einführung Turingmaschine}
\begin{parts}

% Aufgabe 3a
\part Im Binärsystem ist die Multiklikation mit 2 nichts anderes als eine Verschiebung der Bits um eine Stelle nach links.
\begin{enumerate}
\item Gehe nach rechts an das Ende der Zahl.
\item Schreibe eine '0' rechts neben die Zahl.
\item Gehe nach links bis zum Anfang der Zahl und wechsle in den Endzustand.
\end{enumerate}

% Aufgabe 3b
\part Im Binärsystem ist die Multiklikation mit 4 eine Linksverschiebung um 2 Stellen. Entweder man führt die Turingmaschine aus (a) zweimal hintereinander aus oder man modifiziert die Turingmaschine aus (a) und führt den zweiten Schritt zweimal hintereinander aus.\\\\
\end{parts}

%
% Aufgabe 4
%
\titledquestion{Mengen, Funktionen und Relationen}
\begin{parts}

% Aufgabe 4a
\part 
\begin{enumerate}
\item $\{ n \in \mathbb{N} \mid n \mod 4 = 0  \wedge  n \mod 100 = 0 \implies n \mod 400 = 0 \} $ (Schaltjahre)
\item $\{(n,m) \in \mathbb{Z}  \times \mathbb{Z} \mid m \not= 0 \} $ \\
\end{enumerate}

% Aufgabe 4b
\part  $\{1,2,3\} \cup \{x,z\} = \{1,2,3,x,z\}$ \\
$\{1,2,3\} \cap \{x,z\} = \varnothing$ \\
$\{1,2,3\} \setminus \{x,z\} = \{1,2,3\}$ \\
$\{1,2,3\} \setminus \{2\} = \{1,3\}$ \\
$\{1,2,3\} \times \{x,z\} = \{(1,x),(1,z),(2,x),(2,z),(3,x),(3,z)\}$ \\
$\mathcal{P} (\{\varnothing\}) = \{\varnothing,\{\varnothing\}\}$ \\
$\mathcal{P} (\mathcal{P} (\varnothing)) = \{\varnothing,\{\varnothing\},\{\{\varnothing\}\},\{\varnothing,\{\varnothing\}\}\}$ \\

% Aufgabe 4c
\part 
\begin{enumerate}
\item $R_f = \{(x,y) \in X \times Y \mid  f(x) = y\}$
\item Eine Funktion $f_R : X \rightarrow Y$ legt für jedes $ x \in X$ genau ein $y \in Y$ fest. Es gibt allerdings Relationen, wo mehrere y einem x zugewiesen werden:  \\ $R_f =  \{(x,y) \in X \times Y \mid x^2 + y^2 = 2^2\} = \{\dots ,(2,2),(2,-2),\dots \} $
\item Es ist möglich jede Relation $R$ als eine Funktion $g_R : X \rightarrow \mathcal{P}(Y)$ darzustellen, da durch eine Potenzmenge die Probleme der mehrfachen Zuweisung ($x \in X $ wird auf eine Menge mit Mächtigkeit $> 1$ abgebildet) und der fehlenden Zuweisung ($x \in X$ wird auf die leere Menge $\varnothing$ abgebildet) gelöst werden.
\end{enumerate}
\end{parts}

\end{questions}
\end{document}