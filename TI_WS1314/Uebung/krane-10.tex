\documentclass[a4]{exam}
\usepackage[utf8]{inputenc}
\usepackage{amsmath}
\usepackage{amssymb}
\usepackage{algorithm}
\usepackage{algorithmicx}
\usepackage{algpseudocode}

\renewcommand{\author}{Michael Krane}
\newcommand{\gruppe}{Gruppe E2}
\newcommand{\assignment}{Übungsblatt 10}
\newcommand{\class}{BeKo 2013/14}

\pagestyle{headandfoot}
\firstpageheadrule
\runningheadrule
\header{\author}{\gruppe : \assignment}{\class}
\firstpagefooter{}{}{Seite \thepage\ von \numpages}
\runningfooter{}{}{Seite \thepage\ von \numpages}

\qformat{\underline{\textbf{Aufgabe \thequestion}} \thequestiontitle\hfill }

\begin{document}
\begin{questions}
\setcounter{question}{33}

%
% Aufgabe 34
%
\titledquestion {Eigenschaften von Turingmaschinen}
\begin{parts}

% Aufgabe 34a
\part  Entscheidbar. Es ist möglich die TM für 100 Schritte zu simulieren und dann zuüberprüfungen ob die TM sich in einem Endzustand befindet oder nicht.

% Aufgabe 34b
\part Entscheidbar. Man kann die TM simulieren und dabei die durchlaufenen Konfigurationen speichern. Dann tritt irgendwann einer der drei möglichen Fälle ein :
\begin{itemize}
  \item Die Konfiguration der TM ist länger als 100 Zellen. $\Rightarrow$ Abbruch, Aussgabe falsch.
  \item Die TM terminiert mit einer Konfiguration $< 100$ Zellen. $\Rightarrow$ Aussage korrekt.
  \item Die TM terminiert nicht (= eine Konfiguration kommt mehrmals vor), aber alle Konfigurationen sind $< 100$ Zellen $\Rightarrow$ Abbruch. Aussage korrekt.
\end{itemize}

% Aufgabe 34c
\part Angenommen das Problem wäre entscheidbar, dann könnte man jede TM so umbauen das sie, beim Wechsel in den Endzustand ein "a" auf das Band schreibt.
Mit einem solchen Umbau könnte man dann das Halteproblem entscheiden. Die TM hält mit einer Eingabe w an  $\Leftrightarrow $ Die TM schreibt ein "a" auf das Band.
Da das Halteproblem allerdings nicht entscheidbar ist, für diese Annahme zu einem Wiederspruch, also ist die Eigenschaft nicht entscheidbar.

\end{parts}

%
% Aufgabe 35
%
\titledquestion {Eine nicht entscheidbare Sprache}
\begin{parts}

% Aufgabe 35a
\part  Es ist zuzeigen das $\overline{H_0} \le P$: $W_n$ sei die Kodierung einer TM die auf dem leeren Band hält. $w$ sei die Eingabe für $\overline{H_0}$. Dann ist die Vorverarbeitungsfunktion $f(w) = W_n \# w$. \\
Also gilt $\overline{H_0} \le P$ und damit ist P nicht semi-entscheidbar.


% Aufgabe 35b
\part Es ist zuzeigen das $\overline{H_0} \le \overline{P}$: $W_\infty$ sei die Kodierung einer TM die nicht auf dem leeren Band hält. $w$ sei die Eingabe für $\overline{H_0}$. Dann ist die Vorverarbeitungsfunktion $f(w) = w\#W_\infty $. \\
Also gilt $\overline{H_0} \le P$ und damit ist $\overline{P}$ nicht semi-entscheidbar.
\end{parts}

%
% Aufgabe 36
%
\titledquestion {Die Landau-Notation}
\begin{parts}

% Aufgabe 36a
\part  Es sind ein $ N$ und ein $C$ gesucht für die gilt $\forall x > N : \mid f(x)\mid \le C  \mid g(x) \mid $. \\ Versuch mit $N = 1$:
\begin{align}
\mid f(x) \mid &= \mid \sqrt{\mid x\mid} + 10 \mid = \sqrt{x} + 10 \mid\text{  Wegen $N =1$ und $x > N$} \notag \\
&= x^{\frac{1}{2}}  \le C * x^3 \notag \\
&\Leftrightarrow \frac{x^{\frac{1}{2}}}{x^3} + \frac{10}{x^3} \le C \notag \\
&\Leftrightarrow \frac{1}{\sqrt{x^{5}}} + \frac{10}{x^3} \le C \notag \\
&\Leftrightarrow 11 \le C \mid\text{ Maximum der linken Seite ist bei $x = 1$}\notag
\end{align}
Für $x > 1$ und $C \ge 11$ gilt die Ungleichung und damit ist $ f\in O(g)$.

% Aufgabe 36b
\part Da $f(x)$ und $h(x)$ beide in $O(g)$ liegen muss es zwei N  und C geben für die gilt: \\
$\forall x > N_f : \mid f(x) \mid \le C_f \mid g(x) \mid$ \\
$\forall x > N_h : \mid f(x) \mid \le C_h \mid g(x) \mid$ \\
Falls $\varphi \in O(g)$ muss folgendes gelten:
\begin{align}
\forall x > N \text{: } \mid \varphi(x) \mid  &=\notag \\ 
\mid f(x) + h(x) \mid &\le \mid f(x) \mid + \mid h(x) \mid \notag \\ 
&\le C_f * \mid g(x) \mid + C_h * \mid g(x) \mid \notag \\ 
&\le (C_f + C_h) * \mid g(x) \mid \notag \\ 
&= C * \mid g(x) \mid \text{ mit $ C = C_f + C_h$}\notag
\end{align}
Das $N$ für $\varphi (x)$ ist $N = \text{max}(N_f, N_h)$

% Aufgabe 36c
\part Falls: $f \in O(g)  \Rightarrow  \mid f(x) \mid \le C * \mid g(x) \mid$ also
\begin{align}
\forall x > N  \text{:  } &3^x \le C * 2^x \notag \\ 
&\Leftrightarrow \frac{3^x}{2^x} \le C  \notag \\ 
&\Leftrightarrow \left (\frac{3}{2}\right )^x \le C  \notag 
\end{align}
Da $ \frac{3}{2} > 1$ ist das Wachstum nicht beschränkt und deshalb gilt: $f \notin O(g)$

% Aufgabe 36d
\part Sei $x \ge 1$, dann gilt für alle $i \in \{ 1, \dots, n-1 \} : x^i \le x^n$
\begin{align}
\mid f(x)\mid &\text{ =}  \mid \sum\limits_{i=0}^{n} k_i x^i \mid \text{ } \le\text{ } \sum\limits_{i=0}^{n} \mid k_i \mid * \mid x^i \mid \notag \\
&\text{=} \sum\limits_{i=0}^{n} \mid k_i\mid  * x^i    \le \left( \sum\limits_{i=0}^{n}  \mid k_i \mid \right) * x^n          \notag \\
C &= \left( \sum\limits_{i=0}^{n}  \mid k_i \mid \right) \notag
\end{align}
Da es ein N und C gibt gilt $f \in O(x^n)$

\end{parts}

%
% Aufgabe 37
%
\titledquestion {Unäres vs. logarithmisches Kostenmaß}
\begin{parts}

% Aufgabe 37a
\part  unäres Kostenmaß : $O(n * b) = O(n * \log_2(n))$

% Aufgabe 37b
\part  logarithmisches Kostenmaß : $O(n * b) = O(2^b * b)$
\end{parts}

\end{questions}
\end{document}