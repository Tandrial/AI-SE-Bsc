\documentclass[a4]{exam}
\usepackage[utf8]{inputenc}
\usepackage{amsmath}
\usepackage{amssymb}
\usepackage{algorithm}
\usepackage{algorithmicx}
\usepackage{algpseudocode}

\renewcommand{\author}{Michael Krane}
\newcommand{\gruppe}{Gruppe E2}
\newcommand{\assignment}{Übungsblatt 12}
\newcommand{\class}{BeKo 2013/14}

\pagestyle{headandfoot}
\firstpageheadrule
\runningheadrule
\header{\author}{\gruppe : \assignment}{\class}
\firstpagefooter{}{}{Seite \thepage\ von \numpages}
\runningfooter{}{}{Seite \thepage\ von \numpages}

\qformat{\underline{\textbf{Aufgabe \thequestion}} \thequestiontitle\hfill }

\begin{document}
\begin{questions}
\setcounter{question}{40}

%
% Aufgabe 41
%
\titledquestion {Weitere kurze Fragen}
\begin{parts}

% Aufgabe 41a
\part  Falsch, wäre H NP-vollständig also $H \in NP$ müsste H entscheidbar sein.

% Aufgabe 41b
\part Falsch, da $P \subseteq NP$ gitl, würde aus der Aussage sofort $P =NP$ folgen, was allerdings (noch) nicht bekannt ist.

% Aufgabe 41c
\part Richtig, man kann einen Faktor $t \in \mathbb{N}$ raten und dann in polynomialer Zeit überprüfuen ob $2 \le t \ln k$ und ob n durch t teilbar ist.

% Aufgabe 41d
\part Richtig. Wenn man eine Aussagelogischeformel f in KNF negiert, dann ist $\neg f$ in DNF. Es gilt: 
\begin{align}
\text{F hat eine erfüllende Belegung } &\Leftrightarrow \neg f \text{ ist nicht die Tautologie} \notag \\
f \in \text{KNF-SAT} &\Leftrightarrow \neg f \notin \text{DNF-TAU}
\end{align}
$\Rightarrow $ Wäre DNF-TAU in polynomialer Zeit entscheidbar, dann könnte man auch KNF-SAT in polynomialer Zeit entscheiden und damit auch alle anderen Probleme in NP.
Also wäre P = NP, was allerdings (noch) nicht bewiesen ist.
\end{parts}

%
% Aufgabe 42
%
\titledquestion {CLIQUE $\le_p$ INDEPENDENT - SET}
\begin{parts}

% Aufgabe 42a
\part  Sei $G =(V,E)$ ein ungerichteter Graph und k eine Zahl. Wir definieren $f(G,k) = (G', k)$ mit $G'(V,E')$ und $E' = \{(v_1,v_2) \in V \times V \mid v_1 \not= v_2 \wedge (v_1,v_2) \notin E\}$, also die Menge aller Kanten die nicht in E liegen. Diese Funktion f ist total und in polynomialzeit berechenbar.
\begin{align}
(G,k) \in \text{CLIQUE} \Leftrightarrow f(G,k) \in \text{INDEPENDET - SET}
\end{align}
 und damit ist INDEPENDET - SET NP-hart.

% Aufgabe 42b
\part Es ist möglich eine Menge M mit k Knoten zuraten und dann zu überprüfen ob $M \in IS$ ist. Es gibt maximal $\frac{n(n-1)}{2}$ Kanten, also ist das testen in polynomialer Zeit möglich. Zusammen mit dem Ergebnis aus (a) bedeutet das das INDEPENDENT - SET NP-vollständig ist
\end{parts}

%
% Aufgabe 43
%
\titledquestion {VERTEX - COVER}
\begin{parts}

% Aufgabe 43a
\part  Sei $G = (V,E) $ ein ungerichteter Graph mit $n = \mid V\mid$ und $k \in \mathbb{N}$. Wir definieren $f(G,k) = (G,n-k)$. Diese Funktion f ist total und in polynomialzeit berechenbar.\\
Wir müssen zeigen: $(G,k) \in \text{ INDEPENDENT - SET} \Leftrightarrow f(G,k) \in \text{VERTEX - COVER}$\\
"$\Rightarrow$": Sei $(G,k) \in IS$ und sei S ein IS der Größe k. Wir behaupten, dass $ V \backslash S$ ein VC der Größe $n-k$ ist, d. h. dass für alle Kanten $e \in E$ min ein Knoten der Kante in $V \backslash S$ liegt.\\ Angenommen es gäbe eine Kante $e' = \{v_1,v_2\} \in E $ mit $v_1 \notin V \backslash S$ und $v_2 \notin V \backslash S$, d.h. $v_1,v_2 \in S$. Eine solche Kante gibt es nicht, da $S \in IS$ ist.
\\"$\Leftarrow$": Sei $(G,n-k) \in VC$ und C ein VC der Größe $n-k$. Dann ist $V \backslash C $ ein IS der Größe k, d. h. es gibt keine Kannten zwischen ja 2 Knoten aus $V \backslash C$. Angenommen es gäbe eine Kanten mit $ e = {v_1,v_2} \in E$ mit $v_1,v_2 \in V \backslash C$, d.h. $v_1,v_2 \notin C$. Eine solche Kanten gibt es nicht, da C eine Knotenüberdeckung ist.\\
Daraus folgt das VC NP-hart ist.

% Aufgabe 43b
\part Es ist möglich eine Menge M mit k Knoten zuraten und dann zu überprüfen ob $M \in VC$ ist. Es gibt maximal $\frac{n(n-1)}{2}$ Kanten, also ist das testen in polynomialer Zeit möglich. Zusammen mit dem Ergebnis aus (a) bedeutet das das VERTEX - COVER NP-vollständig ist
\end{parts}

\end{questions}
\end{document}